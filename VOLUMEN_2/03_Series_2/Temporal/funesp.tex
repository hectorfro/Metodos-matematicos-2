%% This document created by Scientific Word (R) Version 3.0

\documentclass[spanish,notitlepage,letterpaper,12pt]{article}
\usepackage{graphicx}
\usepackage{amsmath}
\usepackage[spanish]{babel}
\usepackage{amsfonts}
\usepackage{amssymb}
%TCIDATA{OutputFilter=latex2.dll}
%TCIDATA{CSTFile=LaTeX article (bright).cst}
%TCIDATA{Created=Wed Jan 30 04:18:49 2002}
%TCIDATA{LastRevised=Wed Jan 30 05:21:22 2002}
%TCIDATA{<META NAME="GraphicsSave" CONTENT="32">}
%TCIDATA{<META NAME="DocumentShell" CONTENT="Journal Articles\MiArticulo">}
\hyphenation{La-ti-noa-me-ri-ca-na Fa-cul-tad Ren-di-mien-to}
\setlength{\topmargin}{-0.15in}
\setlength{\textheight}{8.2in}
\setlength{\oddsidemargin}{0.0in}
\setlength{\evensidemargin}{0.0in}
\setlength{\textwidth}{6.75in}



\begin{document}


\title{ }



\begin{center}
\textbf{Funci\'{o}n Gamma e Integrales de Probabilidad}

\textbf{La Funci\'{o}n Gamma}
\end{center}

Es la generalizaci\'{o}n del factorial $n!$ el cual s\'{o}lo est\'{a} definido
para enteros, mientras que $\Gamma\left(  z\right)  $ est\'{a} definida para
toda variable compleja $z$ con parte real positiva.

$\Gamma\left(  z\right)  $ se define indistintamente como:
\begin{align*}
\Gamma\left(  z\right)   & =\int_{0}^{\infty}\mathrm{e}^{-t}t^{z-1}%
\mathrm{d}t\equiv\left(  z-1\right)  !\equiv\prod(z-1)\qquad\operatorname{Re}%
z>0\\
\Gamma\left(  z\right)   & =\lim_{n\rightarrow\infty}\frac{1\cdot2\cdot
3\cdot\cdots\cdot n}{z\left(  z+1\right)  \left(  z+2\right)  \cdots\left(
z+n\right)  }n^{z}\\
\frac1{\Gamma\left(  z\right)  }  & =ze^{\gamma z}\prod_{n=1}^{\infty}\left(
1+\frac zn\right)  \mathrm{e}^{-\dfrac zn}%
\end{align*}
donde $n$ es un entero positivo y
\[
\gamma=0.577215664901\cdots
\]
se conoce como la constante de Euler-Mascheroni:

Tambi\'{e}n es frecuente encontrar $\Gamma\left(  z\right)  $ con algunas
variantes cosm\'{e}ticas:
\[
\Gamma\left(  z\right)  =2\int_{0}^{\infty}\mathrm{e}^{-t^{2}}t^{2z-1}%
\mathrm{d}t=\int_{0}^{1}\left[  \ln\left(  \frac1t\right)  \right]
^{z-1}\mathrm{d}t=k^{z}\int_{0}^{\infty}\mathrm{e}^{-kt}t^{z-1}\mathrm{d}t
\]
Para probar la equivalencia de las dos primeras definiciones inventamos las
siguiente funci\'{o}n de dos variables
\[
F(z,n)=\int_{0}^{n}\left(  1-\frac tn\right)  ^{n}t^{z-1}\mathrm{d}%
t\qquad\operatorname{Re}z>0
\]
y como es conocido que
\[
\lim_{n\rightarrow\infty}\left(  1-\frac tn\right)  ^{n}\equiv\mathrm{e}^{-t}
\]
Entonces
\[
\lim_{n\rightarrow\infty}F(z,n)=F(z,\infty)=\int_{0}^{\infty}\mathrm{e}%
^{-t}t^{z-1}\mathrm{d}t\equiv\Gamma\left(  z\right)
\]
Con lo cual queda demostrada la primera de propuestas de Euler.

Para construir la segunda partimos de la misma funci\'{o}n $F(z,n)$ y un
cambio estrat\'{e}gico de variable $u=\frac tn$ .
\[
F(z,n)=n^{z}\int_{0}^{n}\left(  1-u\right)  ^{n}u^{z-1}\mathrm{d}%
u\qquad\operatorname{Re}z>0
\]
Un par de integraciones por partes nos llevan a comprobar
\begin{align*}
F(z,n)  & =n^{z}\left\{  \left.  \left(  1-u\right)  ^{n}\frac{u^{z}}z\right|
_{0}^{1}+\frac nz\int_{0}^{1}\left(  1-u\right)  ^{n-1}u^{z}\mathrm{d}%
u\right\} \\
& =n^{z}\left\{  \left.  \left(  1-u\right)  ^{n-2}u^{z+1}\frac{n(n-1)}%
{z(z+1)}\right|  _{0}^{1}+\frac{n(n-1)}{z(z+1)}\int_{0}^{1}\left(  1-u\right)
^{n-2}u^{z+1}\mathrm{d}u\right\}
\end{align*}
que el primer t\'{e}rmino se anula siempre. Repitiendo el proceso $n$ veces
\begin{align*}
F(z,n)  & =n^{z}\left\{  \frac{n(n-1)(n-2)(n-3)\cdots3\cdot2\cdot
1}{z(z+1)(z+2)(z+3)\cdots(z+n-1)}\right\}  \int_{0}^{1}u^{z+n-1}\mathrm{d}u\\
& =n^{z}\left\{  \frac{n(n-1)(n-2)(n-3)\cdots3\cdot2\cdot1}%
{z(z+1)(z+2)(z+3)\cdots(z+n)}\right\}
\end{align*}
Una vez m\'{a}s, haciendo
\[
\lim_{n\rightarrow\infty}F(z,n)=F(z,\infty)=\lim_{n\rightarrow\infty}%
n^{z}\left\{  \frac{n(n-1)(n-2)(n-3)\cdots3\cdot2\cdot1}%
{z(z+1)(z+2)(z+3)\cdots(z+n)}\right\}  \equiv\Gamma\left(  z\right)
\]
Se completa la equivalencia para la primera y segunda definiciones de Euler.

En particular, de la primera de las definiciones se tiene por integraci\'{o}n
directa
\begin{align*}
\Gamma\left(  1\right)   & =\int_{0}^{\infty}\mathrm{e}^{-t}\mathrm{d}t=1\\
\Gamma\left(  \frac12\right)   & =\int_{0}^{\infty}\mathrm{e}^{-t}%
t^{-1/2}\mathrm{d}t=\int_{0}^{\infty}\mathrm{e}^{-u^{2}}\mathrm{d}u=\sqrt{\pi}%
\end{align*}
mientras que de la segunda, si $z=n$ $=1,2,3,\cdots,$ se obtiene
\begin{align*}
\Gamma\left(  n+1\right)   & =n!\\
\Gamma\left(  n+\frac12\right)   & =\frac{1\cdot3\cdot5\cdot\cdots
(2n-1)}{2^{n}}\sqrt{\pi}%
\end{align*}

Finalmente la tercera de las definiciones de la funci\'{o}n $\Gamma\left(
z\right)  $ viene expresada en t\'{e}rmino de un producto infinito
(Weierstrass). Este puede demostrarse partiendo de la segunda definici\'{o}n
de Euler
\begin{align*}
\Gamma\left(  z\right)   & =\lim_{n\rightarrow\infty}\frac{1\cdot2\cdot
3\cdot\cdots\cdot n}{z\left(  z+1\right)  \left(  z+2\right)  \cdots\left(
z+n\right)  }n^{z}\\
& =\lim_{n\rightarrow\infty}\frac1z\prod_{m=1}^{n}\left(  \frac m{m+z}\right)
n^{z}=\lim_{n\rightarrow\infty}\frac1z\prod_{m=1}^{n}\left(  1+\frac
zm\right)  ^{-1}n^{z}%
\end{align*}
Por lo tanto
\[
\frac1{\Gamma\left(  z\right)  }=z\lim_{n\rightarrow\infty}\prod_{m=1}%
^{n}\left(  1+\frac zm\right)  \mathrm{e}^{-z\ln n}
\]
Ahora bien, multiplicando y dividiendo por
\[
\prod_{m=1}^{n}\mathrm{e}^{z/m}=\mathrm{e}^{z\left(  \sum_{m=1}^{n}%
\frac1m\right)  }
\]
nos queda
\[
\frac1{\Gamma\left(  z\right)  }=z\left\{  \lim_{n\rightarrow\infty}%
\mathrm{e}^{z\left(  \left(  \sum_{m=1}^{n}\frac1m\right)  -\ln n\right)
}\right\}  \left\{  \lim_{n\rightarrow\infty}\prod_{m=1}^{n}\left(  1+\frac
zm\right)  \mathrm{e}^{-z/m}\right\}
\]
Donde, la serie exponente del primero de los t\'{e}rminos converge a un valor
constante y cual ha quedado bautizado como la constante de
\textit{Euler-Mascheroni}
\begin{align*}
\gamma & =\lim_{n\rightarrow\infty}\left\{  1+\frac12+\frac13+\frac
14+\cdots\frac1n-\ln n\right\}  =\lim_{n\rightarrow\infty}\left\{  \left(
\sum_{m=1}^{n}\frac1m\right)  -\ln n\right\} \\
\gamma & =0.5772156649015328606065112\cdots
\end{align*}
Con lo cual queda demostrada la tercera de las propuestas para expresar la
Funci\'{o}n Gamma
\[
\frac1{\Gamma\left(  z\right)  }=ze^{\gamma z}\prod_{n=1}^{\infty}\left(
1+\frac zn\right)  \mathrm{e}^{-\dfrac zn}
\]

Es f\'{a}cil comprobar las siguientes propiedades
\begin{align*}
\Gamma\left(  z+1\right)   & =z\ \Gamma\left(  z\right) \\
\Gamma\left(  z\right)  \Gamma\left(  1-z\right)   & =\int_{0}^{\infty}%
\frac{x^{z-1}\mathrm{d}x}{\left(  1+x\right)  }=\frac\pi{\operatorname*{sen}%
\pi z}\\
2^{2z-1}\Gamma\left(  z\right)  \Gamma\left(  z+\frac12\right)   & =\sqrt{\pi
}\Gamma\left(  2z\right)
\end{align*}
La primera de ellas (la relaci\'{o}n de recurrencia) es trivial y se obtiene
integrando por partes la definici\'{o}n integral de Euler.
\[
\Gamma\left(  z+1\right)  =\int_{0}^{\infty}\mathrm{e}^{-t}t^{z}%
\mathrm{d}t=\left.  z\ \mathrm{e}^{-t}t^{z-1}\right|  _{0}^{\infty}%
+z\ \int_{0}^{\infty}\mathrm{e}^{-t}t^{z-1}\mathrm{d}t=z\Gamma\left(
z\right)
\]
El primer sumando de la integraci\'{o}n por partes se anula siempre. Esta
propiedad es v\'{a}lida $\forall z$ con $z\neq0,-1,-2,\cdots$ .

La segunda de las propiedades (f\'{o}rmula de reflexi\'{o}n) se comprueba
tambi\'{e}n partiendo de definici\'{o}n integral de Euler con el siguiente
cambio de variable $t=u^{2}$.
\begin{align*}
\Gamma\left(  z\right)  \Gamma\left(  1-z\right)   & =2\int_{0}^{\infty
}\mathrm{e}^{-u^{2}}u^{2z-1}\mathrm{d}u\ 2\int_{0}^{\infty}\mathrm{e}^{-v^{2}%
}v^{1-2z}\mathrm{d}v\\
& =4\iint_{0}^{\infty}\mathrm{e}^{-\left(  u^{2}+v^{2}\right)  }\left(  \frac
uv\right)  ^{2z-1}\mathrm{d}u\mathrm{d}v
\end{align*}
si ahora hacemos $u=\rho\cos\varphi$ y $v=\rho\operatorname*{sen}\varphi$, la
integral anterior queda como
\begin{align*}
\Gamma\left(  z\right)  \Gamma\left(  1-z\right)   & =4\int_{0}^{\infty}%
\rho\mathrm{e}^{-\rho^{2}}\ \mathrm{d}\rho\ \int_{0}^{\pi/2}\cot^{2z-1}%
\varphi\mathrm{d}\varphi\\
& =4\cdot\frac12\int_{0}^{\pi/2}\cot^{2z-1}\varphi\mathrm{d}\varphi
\end{align*}
Finalmente, si
\[
\varphi=\operatorname*{arccot}\sqrt{x};\qquad\mathrm{d}\varphi=\frac
{-\mathrm{d}x}{2\sqrt{x}\left(  1+x\right)  }
\]
nos queda
\[
\Gamma\left(  z\right)  \Gamma\left(  1-z\right)  =\int_{0}^{\infty}%
\frac{x^{z-1}\mathrm{d}x}{\left(  1+x\right)  }=\frac\pi{\operatorname*{sen}%
\pi z}
\]
Es inmediato volver a comprobar
\[
\Gamma\left(  \frac12\right)  =\sqrt{\pi}
\]
Del mismo modo, si utilizamos adem\'{a}s la relaci\'{o}n de recurrencia
encontramos
\[
\Gamma\left(  z\right)  \Gamma\left(  -z\right)  =\frac\pi
{-z\operatorname*{sen}\pi z}
\]

La\textit{\ f\'{o}rmula de duplicaci\'{o}n} y puede comprobarse partiendo de
la definici\'{o}n del l\'{\i}mite de Euler, as\'{\i}
\[
\frac{2^{2z-1}\Gamma\left(  z\right)  \Gamma\left(  z+\frac12\right)  }%
{\Gamma\left(  2z\right)  }=\sqrt{\pi}
\]
Hay que hacer notar que en el numerador sustituimos directamente las
expresiones para del l\'{\i}mite de Euler y en la del denominador,
adicionalmente sustituimos $n$ por $2n$
\[
\Gamma\left(  2z\right)  =\lim_{n\rightarrow\infty}\frac{1\cdot2\cdot
3\cdot\cdots\cdot n}{2z\left(  2z+1\right)  \cdots\left(  2z+n\right)  }%
n^{2z}=\lim_{n\rightarrow\infty}\frac{1\cdot2\cdot3\cdot\cdots\cdot
2n}{2z\left(  2z+1\right)  \cdots\left(  2z+2n\right)  }\left(  2n\right)
^{2z}
\]
por lo cual se tiene la siguiente expresi\'{o}n dentro del argumento del
l\'{\i}mite
\[
\frac{2^{2z-1}\left(  \dfrac{1\cdot2\cdot3\cdot\cdots\cdot n}{z\left(
z+1\right)  \left(  z+2\right)  \cdots\left(  z+n\right)  }n^{z}\right)
\left(  \dfrac{1\cdot2\cdot3\cdot\cdots\cdot n}{\left(  z+\frac12\right)
\left(  z+\frac32\right)  \cdots\left(  z+\frac12+n\right)  }n^{z+\frac
12}\right)  }{\left(  \dfrac{1\cdot2\cdot3\cdot\cdots\cdot2n}{2z\left(
2z+1\right)  \left(  2z+2\right)  \cdots\left(  2z+2n\right)  }\left(
2n\right)  ^{2z}\right)  }
\]
la cual se reacomoda como
\[
\lim_{n\rightarrow\infty}\frac{2^{2z-1}\left(  n!\right)  ^{2}2z\left(
2z+1\right)  \left(  2z+2\right)  \cdots\left(  2z+2n\right)  }{\left(
2n\right)  !\ z\left(  z+\frac12\right)  \left(  z+1\right)  \left(
z+\frac32\right)  \left(  z+2\right)  \cdots\left(  z+\frac12+n\right)
\left(  z+n\right)  }\cdot\frac{n^{2z+\frac12}}{\left(  2n\right)  ^{2z}}
\]
y
\[
\lim_{n\rightarrow\infty}\frac{z\left(  z+\frac12\right)  \left(  z+1\right)
\left(  z+\frac32\right)  \left(  z+2\right)  \cdots\left(  z+\frac n2\right)
\left(  2^{n-1}\right)  }{\ z\left(  z+\frac12\right)  \left(  z+1\right)
\left(  z+\frac32\right)  \left(  z+2\right)  \cdots\left(  z+\frac
12+n\right)  \left(  z+n\right)  }\cdot\frac{2^{2z-1}\left(  n!\right)  ^{2}%
}{\left(  2n\right)  !}\cdot\frac{n^{z+\frac12}}{2^{2z}n^{2z}}
\]
Entonces
\[
\frac{2^{2z-1}\Gamma\left(  z\right)  \Gamma\left(  z+\frac12\right)  }%
{\Gamma\left(  2z\right)  }=\lim_{n\rightarrow\infty}\frac{\left(
2^{n-2}\right)  \left(  n!\right)  ^{2}\sqrt{n}}{\left(  2n\right)  !}
\]
por lo cual se deduce que el valor de lado izquierdo de la ecuaci\'{o}n es
independiente del valor de $z$ por lo tanto es el mismo valor para cualquier
$z$ y lo evaluamos para $z=\frac12$
\[
\frac{2^{2z-1}\Gamma\left(  z\right)  \Gamma\left(  z+\frac12\right)  }%
{\Gamma\left(  2z\right)  }=\Gamma\left(  \frac12\right)  =\sqrt{\pi}
\]
con lo cual queda comprobada la f\'{o}rmula de duplicaci\'{o}n.

Otras propiedades que van quedar como curiosidad y sin demostraci\'{o}n son:
\begin{align*}
\Gamma\left(  nz\right)   & =\left(  2\pi\right)  ^{\left(  1-n\right)
/2}n^{nz-\frac12}\prod_{k=0}^{n-1}\left(  z+\frac kn\right) \\
\binom zw  & =\frac{z!}{w!(z-w)!}=\frac{\Gamma\left(  z+1\right)  }%
{\Gamma\left(  w+1\right)  \Gamma\left(  z-w+1\right)  }%
\end{align*}
A partir de $\Gamma\left(  z\right)  $ se definen otras funciones especiales,
las cuales se expresan conjuntamente con sus propiedades como

\begin{center}
\textbf{La Funciones Digamma y Poligamma},
\end{center}

Para evitar tratar con derivadas de los factoriales es costumbre trabajar con
sus derivadas logar\'{\i}tmicas. A partir de la segunda definici\'{o}n
\begin{align*}
\Gamma\left(  z+1\right)   & =z!=\lim_{n\rightarrow\infty}\frac{1\cdot
2\cdot3\cdot\cdots\cdot n}{\left(  z+1\right)  \left(  z+2\right)
\cdots\left(  z+n\right)  }n^{z}\\
\ln\left(  z!\right)   & =\ln\left(  \lim_{n\rightarrow\infty}\frac
{1\cdot2\cdot3\cdot\cdots\cdot n}{\left(  z+1\right)  \left(  z+2\right)
\cdots\left(  z+n\right)  }n^{z}\right) \\
& =\lim_{n\rightarrow\infty}\left(  \ln\left(  n!\right)  +z\ln n-\ln\left(
z+1\right)  -\ln\left(  z+2\right)  -\cdots-\ln\left(  z+n\right)  \right)
\end{align*}
ahora derivando,
\[
\frac{\mathrm{d}}{\mathrm{d}z}\ln\left(  z!\right)  \equiv\mathbf{F}%
(z)=\lim_{n\rightarrow\infty}\left(  \ln n-\frac1{\left(  z+1\right)  }%
-\frac1{\left(  z+2\right)  }-\cdots-\frac1{\left(  z+n\right)  }\right)
\]
y finalmente acomodando, para llegar a la definici\'{o}n m\'{a}s conocida
\[
\mathbf{F}(z)=-\gamma-\sum_{n=1}^{\infty}\left(  \frac1{\left(  z+n\right)
}-\frac1n\right)
\]
Tambi\'{e}n se le conoce como funci\'{o}n Psi
\[
\psi(z)=\frac{\Gamma^{\prime}\left(  z\right)  }{\Gamma\left(  z\right)
}=\frac{\mathrm{d}}{\mathrm{d}z}\ln\left(  \Gamma\left(  z\right)  \right)
\equiv\mathbf{F}(z-1)=\frac{\mathrm{d}}{\mathrm{d}z}\ln\left(  \left(
z-1\right)  !\right)
\]
con las siguientes propiedades
\begin{align*}
\psi(z+1)  & =\frac1z+\psi(z)\\
\psi(z-1)-\psi(z)  & =\pi\cot\pi z\\
\psi(z)+\psi\left(  z+\frac12\right)  +2\ln2  & =2\psi(2z)
\end{align*}
De donde se pueden deducir
\[
\psi(1)=\Gamma^{\prime}\left(  1\right)  =\gamma
\]
La funci\'{o}n $\psi(z)$ puede ser expresada en t\'{e}rminos de integrales
definidas, para ello notamos que
\[
\Gamma^{\prime}\left(  z\right)  =\int_{0}^{\infty}\mathrm{e}^{-t}t^{z-1}\ln
t\ \mathrm{d}t
\]
y sustituyendo la identidad de Frullani
\[
\ln t=\int_{0}^{\infty}\frac{\mathrm{e}^{-x}-\mathrm{e}^{-xt}}x\ \mathrm{d}x
\]
tendremos
\begin{align*}
\Gamma^{\prime}\left(  z\right)   & =\int_{0}^{\infty}\mathrm{e}^{-t}%
t^{z-1}\int_{0}^{\infty}\frac{\mathrm{e}^{-x}-\mathrm{e}^{-xt}}x\ \mathrm{d}%
x\ \mathrm{d}t\\
& =\int_{0}^{\infty}\frac{\mathrm{d}x}x\int_{0}^{\infty}\left(  \mathrm{e}%
^{-x}-\mathrm{e}^{-xt}\right)  \mathrm{e}^{-t}t^{z-1}\mathrm{d}t\\
& =\int_{0}^{\infty}\frac{\mathrm{d}x}x\mathrm{e}^{-x}\int_{0}^{\infty
}\mathrm{e}^{-t}t^{z-1}\mathrm{d}t-\int_{0}^{\infty}\frac{\mathrm{d}x}%
x\int_{0}^{\infty}\mathrm{e}^{-t\left(  x+1\right)  }t^{z-1}\mathrm{d}t\\
& =\Gamma\left(  z\right)  \int_{0}^{\infty}\frac{\mathrm{d}x}x\left[
\mathrm{e}^{-x}-\left(  x+1\right)  ^{-z}\right]
\end{align*}
ya que $\Gamma\left(  z\right)  =k^{z}\int_{0}^{\infty}\mathrm{e}^{-kt}%
t^{z-1}\mathrm{d}t $ y por lo tanto
\[
\psi(z)=\int_{0}^{\infty}\frac{\mathrm{d}x}x\left[  \mathrm{e}^{-x}-\left(
x+1\right)  ^{-z}\right]
\]
Tambi\'{e}n daremos (sin demostraci\'{o}n) otras expresiones
\begin{align*}
\psi(z)  & =\int_{0}^{\infty}\left(  \frac{\mathrm{e}^{-t}}t-\frac
{\mathrm{e}^{-tz}}{1-\mathrm{e}^{-t}}\right)  \mathrm{d}t\\
\psi(z)  & =-\gamma+\int_{0}^{1}\frac{1-x^{z-1}}{1-x}\mathrm{d}x
\end{align*}

La Funci\'{o}n Poligamma se obtiene derivando en forma repetida la Funci\'{o}n
Digamma
\[
\psi^{(m)}(z+1)=\mathbf{F}^{(m)}(z)=\frac{\mathrm{d}^{m}}{\mathrm{d}z^{m}%
}\mathbf{F}(z)=(-1)^{m+1}m!\sum_{n=1}^{\infty}\frac1{\left(  z+n\right)
^{m+1}}\qquad m=1,2,3\cdots
\]
y cuya serie puede ser expresada en t\'{e}rminos de la funci\'{o}n Zeta de
Riemman
\[
\zeta(m)\equiv\sum_{n=1}^{\infty}\frac1{n^{m}}
\]
como
\[
\mathbf{F}^{(m)}(0)=-1)^{m+1}m!\zeta(m+1)
\]
de esta forma es posible desarrollar en serie de Maclaurin
\[
\ln(n!)=-\gamma+\frac{z^{2}}2\zeta(2)-\frac{z^{3}}3\zeta(3)+\cdots+\left(
-1\right)  ^{n}\frac{z^{n}}n\zeta(n)+\cdots
\]

\begin{center}
\textbf{La Aproximaci\'{o}n de Stirling}
\end{center}

El comportamiento asint\'{o}tico de las funciones especiales ser\'{a} tratado
en una clase aparte. Pero la importancia de la Aproximaci\'{o}n de Stirling
obliga a que se trate en este punto. Supongamos que consideramos el caso
$z\equiv x\in\Re.$ Por lo cual estamos interesados en el caso $x\gg1. $
Partimos de
\[
\Gamma\left(  x\right)  =\frac1x\Gamma\left(  x+1\right)  =\frac1x\int
_{0}^{\infty}\mathrm{e}^{-t}t^{x}\mathrm{d}t=\frac1x\int_{0}^{\infty
}\mathrm{e}^{-t+x\ln t}\mathrm{d}t
\]
haciendo $t=xu$ tenemos que
\[
\Gamma\left(  x\right)  =x^{x}\int_{0}^{\infty}\mathrm{e}^{-x\left(  u-\ln
u\right)  }\mathrm{d}u
\]
Ahora bien, el integrando tendr\'{a} su m\'{a}ximo en $u=1$ donde la
exponencial tiene su m\'{\i}nimo y es entorno a ese punto que desarrollar\'{a}
en series de Taylor
\[
u-\ln u=1+\frac12\left(  u-1\right)  ^{2}-\frac13\left(  u-1\right)
^{3}+\frac14\left(  u-1\right)  ^{4}+\cdots
\]
por lo cual
\[
\Gamma\left(  x\right)  =x^{x}\int_{0}^{\infty}\mathrm{e}^{-x\left(  u-\ln
u\right)  }\mathrm{d}u\approx x^{x}\int_{0}^{\infty}\mathrm{d}u\ \mathrm{e}%
^{-x\left(  1+\frac12\left(  u-1\right)  ^{2}-\frac13\left(  u-1\right)
^{3}+\cdots\right)  }\mathrm{d}u
\]
Otro cambio de variable $v=$ $\sqrt{x}\left(  u-1\right)  $ nos lleva
\[
\Gamma\left(  x\right)  \approx\frac{x^{x}\mathrm{e}^{-x}}{\sqrt{x}}%
\int_{-\sqrt{x}}^{\infty}\mathrm{d}v\mathrm{e}^{-\frac12v^{2}}\exp\left(
\frac1{3\sqrt{x}}v^{3}-\frac1{4x}v^{4}+\frac1{5x^{\frac32}}v^{5}%
-\cdots\right)  \
\]
Para valores $x\gg1$ se expande, en series de Taylor los exponenciales que
contengan t\'{e}rminos $\frac1{\sqrt{x}}$%
\begin{align*}
\Gamma\left(  x\right)   & \approx\frac{x^{x}\mathrm{e}^{-x}}{\sqrt{x}}%
\int_{-\infty}^{\infty}\mathrm{d}v\mathrm{e}^{-\frac12v^{2}}\left\{  1+\left(
\frac1{3\sqrt{x}}v^{3}-\frac1{4x}v^{4}+\frac1{5x^{\frac32}}v^{5}%
-\cdots\right)  +\right. \\
& +\frac1{2!}\left(  \frac1{3\sqrt{x}}v^{3}-\frac1{4x}v^{4}+\frac
1{5x^{\frac32}}v^{5}-\cdots\right)  ^{2}+\\
& \left.  +\frac1{3!}\left(  \frac1{3\sqrt{x}}v^{3}-\frac1{4x}v^{4}%
+\frac1{5x^{\frac32}}v^{5}-\cdots\right)  ^{3}+\cdots\right\}
\end{align*}
Finalmente, utilizando que
\[
\int_{-\infty}^{\infty}\mathrm{d}v\mathrm{e}^{-\frac12v^{2}}v^{n}=\left\{
\begin{array}
[c]{l}%
\sqrt{2\pi}\qquad\\
\sqrt{2\pi}\cdot1\cdot3\cdot5\cdot\cdots(n-1)\\
0
\end{array}
\quad%
\begin{array}
[c]{l}%
n=0\\
n=2k\\
n=2k-1
\end{array}
\right.
\]
e integrando t\'{e}rmino a t\'{e}rmino, tendremos que
\[
\Gamma\left(  x\right)  \approx\sqrt{\frac{2\pi}x}x^{x}\mathrm{e}^{-x}\left\{
1+\frac1{12x}+\frac1{288\ x^{2}}+\cdots\right\}
\]

\begin{center}
\textbf{La funci\'{o}n Beta}
\begin{align*}
B(x,y)  & =\int_{0}^{1}t^{x-1}\left(  1-t\right)  ^{y-1}\mathrm{d}%
t\qquad\operatorname{Re}x>0\ \wedge\ \operatorname{Re}y>0\ \\
B(x,y)  & =\frac{\Gamma\left(  x\right)  \Gamma\left(  y\right)  }%
{\Gamma\left(  x+y\right)  }%
\end{align*}
\end{center}

\textbf{La Funci\'{o}n Integral de Probabilidad}\newline La funci\'{o}n
Integral de Probabilidad para una variable compleja arbitraria $z$ como
\[
\Phi(z)=\frac2{\sqrt{\pi}}\int_{0}^{z}\mathrm{e}^{-t^{2}}\mathrm{d}t
\]
Obviamente $\Phi(0)=0$ y $\Phi(\infty)=1.$ A partir de esta funci\'{o}n se
define la \textbf{Funci\'{o}n Error y su complemento}

\begin{center}%
\begin{align*}
\operatorname{erf}(z)  & =\int_{0}^{z}\mathrm{e}^{-t^{2}}\mathrm{d}%
t=\frac{\sqrt{\pi}}2\Phi(z)\\
\operatorname{erf}c(z)  & =\int_{z}^{z}\mathrm{e}^{-t^{2}}\mathrm{d}%
t=\frac{\sqrt{\pi}}2\left[  1-\Phi(z)\right]
\end{align*}
\textbf{Funci\'{o}n Gamma Incompleta }$\gamma\left(  z,\alpha\right)
$\textbf{\ y }

\textbf{Funci\'{o}n Gamma Complementaria }$\Gamma\left(  z,\alpha\right)  $
\end{center}%

\begin{align*}
\gamma\left(  z,\alpha\right)    & =\int_{0}^{\alpha}\mathrm{e}^{-t}%
t^{z-1}\mathrm{d}t\\
\Gamma\left(  z,\alpha\right)    & =\int_{\alpha}^{\infty}\mathrm{e}%
^{-t}t^{z-1}\mathrm{d}t
\end{align*}
las cuales claramente cumplen con
\[
\gamma\left(  z,\alpha\right)  +\Gamma\left(  z,\alpha\right)  =\Gamma\left(
z\right)
\]
y resumen
\begin{align*}
\gamma\left(  z+1,\alpha\right)    & =z\gamma\left(  z,\alpha\right)
-\alpha^{z}\mathrm{e}^{-\alpha}\\
\Gamma\left(  z+1,\alpha\right)    & =z\Gamma\left(  z,\alpha\right)
+\alpha^{z}\mathrm{e}^{-\alpha}%
\end{align*}

\newpage

\begin{center}
\textbf{M\'{e}todos Matem\'{a}ticos de la F\'{i}sica}

\textbf{Ecuaciones Diferenciales de Legendre, Laguerre y Hermite}

\textbf{y Series de Polinomios Ortogonales }
\end{center}

\textbf{Polinomios de Legendre}\newline La ecuaci\'{o}n de Legendre
\[
(1-x^{2})\ y^{\prime\prime}-2x\ y^{\prime}+\lambda(\lambda+1)\ y=0
\]
tiene singularidades en $x=\pm1.$ Por lo tanto, todos los $x$ son ordinarios
si $x\in(-1,1).$ En ese intervalo se propone una soluci\'{o}n
\[
y(x)=\sum_{n=0}^{\infty}a_{n}x^{n}
\]
por lo tanto
\begin{align*}
(1-x^{2})\ \sum_{n=2}^{\infty}n(n-1)a_{n}x^{n-2}-  & \\
-2x\ \sum_{n=1}^{\infty}n\ a_{n}x^{n-1}+\lambda(\lambda+1)\sum_{n=0}^{\infty
}a_{n}x^{n}  & =0
\end{align*}
multiplicando y acomodando
\begin{align*}
\sum_{j=0}^{\infty}(j+2)(j+1)a_{j+2}x^{j}-\ \sum_{n=2}^{\infty}n(n-1)a_{n}%
x^{n}-  & \\
-2\ \sum_{n=1}^{\infty}n\ a_{n}x^{n}+\lambda(\lambda+1)\sum_{n=0}^{\infty
}a_{n}x^{n}  & =0
\end{align*}
expandiendo
\begin{align*}
2a_{2}+\lambda(\lambda+1)a_{0}\left\{  (\lambda+2)(\lambda-1)a_{1}%
+(3\cdot2)a_{3}\right\}  x-  & \\
+\ \sum_{n=2}^{\infty}\left\{  (n+2)(n+1)a_{n+2}+(\lambda+n+1)(\lambda
-n)a_{n}\right\}  x^{n}  & =0
\end{align*}
donde hemos utilizado
\[
-n(n-1)-2n+\lambda(\lambda+1)=(\lambda+n+1)(\lambda-n)
\]
por lo tanto
\begin{align*}
a_{2}  & =-\frac{(\lambda+1)\lambda}2\ a_{0}\\
a_{4}  & =\frac{(\lambda+3)(\lambda+1)\lambda(\lambda-2)}{4!}\ a_{0}\\
a_{2n}  & =(-1)^{n}\frac{(\lambda+2n-1)(\lambda+2n-3)\cdots(\lambda
+1)\lambda(\lambda-2)\cdots(\lambda-2n+2)}{(2n)!}\ a_{0}%
\end{align*}
y las potencias impares ser\'{a}n
\begin{align*}
a_{3}  & =-\frac{(\lambda+2)(\lambda-1)}{3!}\ a_{1}\\
a_{5}  & =\frac{(\lambda+4)(\lambda+2)(\lambda-1)(\lambda-3)}{5!}\ a_{1}\\
a_{2n+1}  & =(-1)^{n}\frac{(\lambda+2n)(\lambda+2n-2)\cdots(\lambda
+2)(\lambda-1)\cdots(\lambda-2n+1)}{(2n+1)!}\ a_{1}%
\end{align*}
y su soluci\'{o}n general de la forma
\[
y(x)=a_{0}\ y_{0}(x)+a_{1}y_{1}(x)
\]
con
\begin{align*}
y_{0}(x)  & =1-\frac{(\lambda+1)\lambda}2\ x^{2}+\frac{(\lambda+3)(\lambda
+1)\lambda(\lambda-2)}{4!}\ x^{4}+\cdots\\
y_{1}(x)  & =x-\frac{(\lambda+2)(\lambda-1)}{3!}\ x^{3}+\frac{(\lambda
+4)(\lambda+2)(\lambda-1)(\lambda-3)}{5!}\ x^{5}+\cdots
\end{align*}
si $\lambda=2n$ la soluci\'{o}n es un polinomio de potencias pares y si
$\lambda=2n+1$ es uno de potencias impares

\begin{center}%
\begin{tabular}
[c]{lcc}\hline\hline
$\lambda$ & Ecuaci\'{o}n de Legendre & Soluci\'{o}n\\\hline\hline
$0$ & \multicolumn{1}{r}{$(1-x^{2})\ y^{\prime\prime}-2x\ y^{\prime}=0$} &
\multicolumn{1}{l}{$y_{0}(x)=1$}\\
$1$ & \multicolumn{1}{r}{$(1-x^{2})\ y^{\prime\prime}-2x\ y^{\prime}+2\ y=0$}%
& \multicolumn{1}{l}{$y_{1}(x)=x$}\\
$2$ & \multicolumn{1}{r}{$(1-x^{2})\ y^{\prime\prime}-2x\ y^{\prime}+6\ y=0$}%
& \multicolumn{1}{l}{$y_{0}(x)=1-3x^{2}$}\\
$3$ & \multicolumn{1}{r}{$(1-x^{2})\ y^{\prime\prime}-2x\ y^{\prime}+12\ y=0
$} & \multicolumn{1}{l}{$y_{1}(x)=x-\frac53x^{3}$}\\
$4$ & \multicolumn{1}{r}{$(1-x^{2})\ y^{\prime\prime}-2x\ y^{\prime}+20\ y=0
$} & \multicolumn{1}{l}{$y_{0}(x)=1-10x^{2}+\frac{35}3x^{4}$}\\\hline\hline
\end{tabular}
\end{center}

\textbf{F\'{o}rmula de Rodr\'{\i}guez}\newline Se definen como los polinomios
de Legendre las soluciones a las ecuaciones arriba expuestas para $\lambda$
dados o tambi\'{e}n a partir de la F\'{o}rmula de Rodr\'{\i}guez
\[
P_{n}(x)=\frac1{n!2^{n}}\frac{\mathrm{d}^{n}}{\mathrm{d}x^{n}}(x^{2}%
-1)^{n},\qquad n=0,1,2,.....
\]
con $P_{0}(x)=1.$

\textbf{Ortogonalidad de los Polinomios de Legendre}
\newline 
Como los polinomios de Legendre son soluciones de su ecuaciones
\begin{align*}
(1-x^{2})\ P_{\alpha}(x)^{\prime\prime}-2x\ \ P_{\alpha}(x)^{\prime}%
+\alpha(\alpha+1)\ P_{\alpha}(x)  & =0\\
(1-x^{2})\ P_{\beta}(x)^{\prime\prime}-2x\ \ P_{\beta}(x)^{\prime}+\beta
(\beta+1)\ P_{\beta}(x)  & =0
\end{align*}
Acomodando y restando ambas ecuaciones
\begin{align*}
(1-x^{2})\left\{  \ P_{\beta}(x)P_{\alpha}(x)^{\prime\prime}-\ P_{\alpha
}(x)P_{\beta}(x)^{\prime\prime}\right\}  \   & -\\
-2x\ \left\{  P_{\beta}(x)P_{\alpha}(x)^{\prime}-P_{\alpha}(x)P_{\beta
}(x)^{\prime}\right\}    & +\\
+\left\{  \alpha(\alpha+1)-\beta(\beta+1)\right\}  P_{\beta}(x)P_{\alpha}(x)
& =0
\end{align*}
el primer t\'{e}rmino de la ecuaci\'{o}n puede interpretarse una la derivada
\[
\left[  (1-x^{2})\left\{  \ P_{\beta}(x)P_{\alpha}(x)^{\prime}-\ P_{\alpha
}(x)P_{\beta}(x)^{\prime}\right\}  \right]  ^{\prime}%
\]
por lo tanto al integrar
\begin{align*}
\left.  (1-x^{2})\left\{  \ P_{\beta}(x)P_{\alpha}(x)^{\prime}-\ P_{\alpha
}(x)P_{\beta}(x)^{\prime}\right\}  \right|  _{-1}^{1}  & \\
\left\{  \alpha(\alpha+1)-\beta(\beta+1)\right\}  \int_{-1}^{1}P_{\alpha
}(x)P_{\beta}(x)\mathrm{d}x  & =0
\end{align*}
El primer t\'{e}rmino de la ecuaci\'{o}n se anula en los extremos y es
f\'{a}cil comprobar que los polinomios de Legendre $|\mathbf{P}_{\alpha
}\rangle=P_{\alpha}(x)$ son mutuamente ortogonales con un producto interno
definido como
\[
\langle\mathbf{P}_{\alpha}|\mathbf{P}_{\beta}\rangle=\int_{-1}^{1}P_{\alpha
}(x)P_{\beta}(x)\mathrm{d}x\propto\delta_{\alpha\beta}%
\]

\textbf{Relaci\'{o}n de Recurrencia}

Conocido esto se puede generar una relaci\'{o}n de recurrencia. Supongamos que
conocemos todos los polinomios de Legendre hasta $P_{n}(x)$ y queremos generar
el pr\'{o}ximo. Obviamente el ese polinomio ser\'{a} de grado $n+1$ y nos
plantemos generarlo a partir de $xP_{n}(x)$ as\'{i} como los estos polinomios
son base del espacio de funciones, entonces
\[
xP_{n}(x)=|x\mathbf{P}_{n}\rangle=\sum_{k=0}^{n+1}\frac{\langle\mathbf{P}%
_{k}|x\mathbf{P}_{n}\rangle}{\langle\mathbf{P}_{k}|\mathbf{P}_{k}\rangle
}\ |\mathbf{P}_{k}\rangle
\]
en donde
\[
\langle\mathbf{P}_{k}|x\mathbf{P}_{n}\rangle=\langle x\mathbf{P}%
_{k}|\mathbf{P}_{n}\rangle=\int_{-1}^{1}P_{n}(x)xP_{k}(x)\mathrm{d}x=0
\]
para $k<n-1.$ Sobreviven entonces tres t\'{e}rminos
\[
|x\mathbf{P}_{n}\rangle=xP_{n}(x)=\frac{\langle\mathbf{P}_{n-1}|x\mathbf{P}%
_{n}\rangle}{\langle\mathbf{P}_{n-1}|\mathbf{P}_{n-1}\rangle}\ |\mathbf{P}%
_{n-1}\rangle+\frac{\langle\mathbf{P}_{n}|x\mathbf{P}_{n}\rangle}%
{\langle\mathbf{P}_{n}|\mathbf{P}_{n}\rangle}\ |\mathbf{P}_{n}\rangle
+\frac{\langle\mathbf{P}_{n+1}|x\mathbf{P}_{n}\rangle}{\langle\mathbf{P}%
_{n+1}|\mathbf{P}_{n+1}\rangle}\ |\mathbf{P}_{n+1}\rangle
\]
y dado que
\[
\langle\mathbf{P}_{n}|x\mathbf{P}_{n}\rangle=\int_{-1}^{1}P_{n}(x)xP_{n}%
(x)\mathrm{d}x=\int_{-1}^{1}xP_{n}^{2}(x)\mathrm{d}x\quad,
\]
es una funci\'{o}n impar, entonces $\langle\mathbf{P}_{n}|x\mathbf{P}%
_{n}\rangle=0.$ Entonces
\[
|x\mathbf{P}_{n}\rangle=xP_{n}(x)=\frac{\langle\mathbf{P}_{n-1}|x\mathbf{P}%
_{n}\rangle}{\langle\mathbf{P}_{n-1}|\mathbf{P}_{n-1}\rangle}\ |\mathbf{P}%
_{n-1}\rangle+\frac{\langle\mathbf{P}_{n+1}|x\mathbf{P}_{n}\rangle}%
{\langle\mathbf{P}_{n+1}|\mathbf{P}_{n+1}\rangle}\ |\mathbf{P}_{n+1}\rangle
\]
Es decir
\[
xP_{n}(x)=AP_{n+1}(x)+BP_{n-1}(x)
\]
desarrollando con la f\'{o}rmula de Rodr\'{i}guez el coeficiente de orden $k$
del lado izquierdo es
\[
\frac{1}{2^{k}k!}2k(2k-1)\cdots\left[  2k-(k-1)\right]  =\frac{(2k)!}%
{2^{k}(k!)^{2}}%
\]
mientras que el primer t\'{e}rmino del lado izquierdo, hasta orden $k-2$ queda
como
\[
\frac{(2k-2)!}{2^{k}(k-2)!(k-1)}%
\]
por lo cual
\[
A=\frac{n+1}{2n+1}%
\]
De igual forma se determina $B$ igualando coeficientes a orden $n-1$ y queda
la relaci\'{o}n de recurrencia:
\[
\left(  n+1\right)  P_{n+1}(x)=\ \left(  2n+1\right)  xP_{n}(x)-nP_{n-1}(x)
\]

\textbf{Norma de los Polinomios de Legendre}\newline Conociendo que la
ortogonalidad de los polinomios de Legendre y la relaci\'{o}n de recurrencia,
procedemos encontrar el valor de su norma
\[
\left\|  \mathbf{P}_{n}\right\|  ^{2}=\langle\mathbf{P}_{n}|\mathbf{P}%
_{n}\rangle=\int_{-1}^{1}P_{n}^{2}(x)\mathrm{d}x\check{=}\frac2{2n+1}
\]
De la relaci\'{o}n de recurrencia
\begin{align*}
\left(  2n+1\right)  P_{n}(x)nP_{n}(x)  & =\left(  2n+1\right)  P_{n}%
(x)\left[  \left(  2n-1\right)  \ xP_{n-1}(x)-\left(  n-1\right)
P_{n-2}(x)\right] \\
\left(  2n-1\right)  P_{n-1}(x)\left(  n+1\right)  P_{n+1}(x)  & =\left(
2n-1\right)  P_{n-1}(x)\ \left[  \left(  2n+1\right)  xP_{n}(x)-nP_{n-1}%
(x)\right]
\end{align*}
restando miembro a miembro obtenemos:
\begin{align*}
\left(  2n+1\right)  P_{n}(x)nP_{n}(x)+\left(  2n+1\right)  \left(
n-1\right)  P_{n}(x)P_{n-2}(x)-\  & \\
-\left(  n+1\right)  \left(  2n-1\right)  P_{n-1}(x)P_{n+1}(x)-\left(
2n-1\right)  nP_{n-1}^{2}(x)  & =0
\end{align*}
integrando y considerando la ortogonalidad
\begin{align*}
\int_{-1}^{1}P_{n}^{2}(x)\mathrm{d}x  & =\frac{2n-1}{2n+1}\int_{-1}^{1}%
P_{n-1}^{2}(x)\mathrm{d}x\\
\int_{-1}^{1}P_{n}^{2}(x)\mathrm{d}x  & =\left(  \frac{2n-1}{2n+1}\right)
\left(  \frac{2n-3}{2n-1}\right)  \int_{-1}^{1}P_{n-2}^{2}(x)\mathrm{d}x\\
\int_{-1}^{1}P_{n}^{2}(x)\mathrm{d}x  & =\left(  \frac{2n-1}{2n+1}\right)
\left(  \frac{2n-3}{2n-1}\right)  \left(  \frac{2n-5}{2n-3}\right)  \int
_{-1}^{1}P_{n-3}^{2}(x)\mathrm{d}x\\
\vdots\qquad & =\qquad\vdots\\
\int_{-1}^{1}P_{n}^{2}(x)\mathrm{d}x  & \check{=}\frac3{2n+1}\int_{-1}%
^{1}P_{1}^{2}(x)\mathrm{d}x\\
\int_{-1}^{1}P_{n}^{2}(x)\mathrm{d}x  & \check{=}\frac2{2n+1}%
\end{align*}

\textbf{Otras propiedades de los polinomios de Legendre}, dignas de ser
mencionadas son

\begin{itemize}
\item $P_{n}(1)=1$ y $P_{n}(-1)=(-1)^{n}$ para todo $n.$

\item $P_{n}(x)$ tiene $n$ ra\'{i}ces en el intervalo $\left(  -1,1\right)  $
Esta propiedad puede apreciarse para los primeros 5 polinomios en la figura \ref{poligendre}
\end{itemize}

\begin{center}%
%TCIMACRO{\FRAME{ftbpFU}{4.5238in}{3.3987in}{0pt}{\Qcb{Polinomios de
%Lengendre}}{\Qlb{poligendre}}{legendre1.jpg}%
%{\special{ language "Scientific Word";  type "GRAPHIC";
%maintain-aspect-ratio TRUE;  display "USEDEF";  valid_file "F";
%width 4.5238in;  height 3.3987in;  depth 0pt;  original-width 9.9895in;
%original-height 7.4893in;  cropleft "0";  croptop "1";  cropright "1";
%cropbottom "0";  filename 'legendre1.JPG';file-properties "XNPEU";}}}%
%BeginExpansion
\begin{figure}
[ptb]
\begin{center}
\includegraphics[
natheight=7.489300in,
natwidth=9.989500in,
height=3.3987in,
width=4.5238in
]%
{legendre1.jpg}%
\caption{Polinomios de Lengendre}%
\label{poligendre}%
\end{center}
\end{figure}
%EndExpansion
\end{center}

\begin{itemize}
\item  Tienen una representaci\'{o}n integral de la forma
\[
P_{n}(x)=\frac1{2\pi}\int_{0}^{\pi}\left[  x+\sqrt{x^{2}-1}\cos\varphi\right]
^{n}\mathrm{d}\varphi
\]

\item  Cambios de variables inmediatos conllevan a ecuaciones diferenciales equivalentes

\begin{itemize}
\item  Forma autoadjunta
\[
\left[  (1-x^{2})\ y^{\prime}\right]  ^{\prime}+\lambda(\lambda+1)\ y=0
\]

\item  En coordenadas esf\'{e}ricas con $u=P_{n}(\cos\theta)$%
\[
\frac1{\operatorname*{sen}\theta}\frac{\mathrm{d}}{\mathrm{d}\theta}\left(
\operatorname*{sen}\theta\ \frac{\mathrm{d}u}{\mathrm{d}\theta}\right)
+\lambda(\lambda+1)u=0
\]

\item  En coordenadas esf\'{e}ricas con $u=\sqrt{\operatorname*{sen}\theta
}P_{n}(\cos\theta)$%
\[
\frac{\mathrm{d}^{2}u}{\mathrm{d}\theta^{2}}+\left[  \left(  \lambda
+\frac12\right)  ^{2}+\frac1{4\operatorname*{sen}^{2}\theta}\right]  u=0
\]
\end{itemize}
\end{itemize}

\textbf{Funci\'{o}n Generatriz de los Polinomios de Legendre}\newline Se puede
encontrar una funci\'{o}n generatriz $\mathcal{P}(t,x)$ de los polinomios de
Legendre:
\[
\mathcal{P}(t,x)=\frac1{\sqrt{1-2xt+t^{2}}}=P_{0}(x)\ +P_{1}(x)\ t+P_{2}%
(x)\ t^{2}+\cdots=\sum_{n=0}^{\infty}P_{n}(x)\ t^{n}
\]
para la cual los $P_{n}(x)$ son los coeficientes de su desarrollo en series de
potencias. Esta serie converge para $\left\|  2xt+t^{2}\right\|  <1.$ Para
demostrar que el desarrollo en serie de la funci\'{o}n $\mathcal{G}(t,x)$
tiene como coeficientes a los $P_{n}(x)$ partimos de que:
\[
\mathcal{P}(t,x)=\frac1{\sqrt{1-2xt+t^{2}}}\quad\Rightarrow\quad\frac
{\partial\mathcal{P}(t,x)}{\partial t}=\frac{t-x}{\left(  1-2xt+t^{2}\right)
^{3/2}}\quad
\]
por lo cual
\[
\left(  t-x\right)  \mathcal{P}(t,x)+\left(  1-2xt+t^{2}\right)
\frac{\partial\mathcal{P}(t,x)}{\partial t}=0
\]
y, consecuentemente
\[
\left(  t-x\right)  \sum_{n=0}^{\infty}P_{n}(x)\ t^{n}+\left(  1-2xt+t^{2}%
\right)  \sum_{n=0}^{\infty}nP_{n}(x)\ t^{n-1}=0\ .
\]
Multiplicando y acomodando queda
\begin{align*}
-x\ P_{0}(x)+\ P_{0}(x)\ t\ +\sum_{n=0}^{\infty}\left(  n+1\right)
P_{n+1}(x)\ t^{n}  & -\\
-\sum_{n=1}^{\infty}\left(  2n+1\right)  \ x\ P_{n}(x)\ t^{n}-\sum
_{n=2}^{\infty}nP_{n-1}(x)\ t^{n}  & =0
\end{align*}
por lo tanto
\begin{align*}
\left[  \underbrace{P_{1}(x)-x\ P_{0}(x)}_{=0}\right]  +\ \left[
\underbrace{2P_{2}(x)-3xP_{1}(x)+P_{0}(x)}_{=0}\right]  \ t\ -  & \\
+\sum_{n=1}^{\infty}\left[  \underbrace{\left(  n+1\right)  P_{n+1}(x)-\left(
2n+1\right)  \ xP_{n}(x)+nP_{n-1}(x)}_{=0}\right]  \ t^{n}  & =0
\end{align*}
El primero de los t\'{e}rminos se cumple siempre por cuanto $P_{0}(x)=1$ y
$P_{1}(x)=x.$ El tercer t\'{e}rmino conforma la relaci\'{o}n de recurrencia
para los polinomios de Legendre. Con esto queda demostrado que el desarrollo
en series de potencias de la funci\'{o}n generatriz, tiene como coeficientes a
los polinomios de Legendre.

\textbf{Un Ejemplo de la F\'{i}sica}\newline En F\'{i}sica el ejemplo claro es
el c\'{a}lculo del potencial electrost\'{a}tico producido por dos cargas
$q_{1}=+q$ y $q_{2}=-q$ separadas por una distancia $2d$ en un punto $P$
cualquiera de un plano $\left(  x,y\right)  $. El potencial en ese punto
gen\'{e}rico viene dado por
\[
V=q\left(  \frac{1}{R^{\prime}}-\frac{1}{R}\right)
\]%

%TCIMACRO{\FRAME{ftbpF}{4.5238in}{3.3987in}{0pt}{}{\Qlb{potencial}%
%}{potencial.jpg}{\special{ language "Scientific Word";  type "GRAPHIC";
%maintain-aspect-ratio TRUE;  display "USEDEF";  valid_file "F";
%width 4.5238in;  height 3.3987in;  depth 0pt;  original-width 9.9895in;
%original-height 7.4893in;  cropleft "0";  croptop "1";  cropright "1";
%cropbottom "0";  filename 'potencial.JPG';file-properties "XNPEU";}}}%
%BeginExpansion
\begin{figure}
[ptb]
\begin{center}
\includegraphics[
natheight=7.489300in,
natwidth=9.989500in,
height=3.3987in,
width=4.5238in
]%
{potencial.jpg}%
\label{potencial}%
\end{center}
\end{figure}
%EndExpansion

Tal y como puede apreciarse de la figura \ref{potencial}
\begin{align*}
\left(  R^{\prime}\right)  ^{2}  & =r^{2}+d^{2}-2r\ d\cos\theta\\
R^{2}  & =r^{2}+d^{2}-2r\ d\cos\left(  \pi-\theta\right)
\end{align*}
por lo cual
\begin{align*}
\frac{1}{R^{\prime}}  & =\frac{1}{r}\left[  1-\left(  2\frac{d}{r}\ \cos
\theta-\left\{  \frac{d}{r}\right\}  ^{2}\right)  \right]  ^{-1/2}\\
\frac{1}{R}  & =\frac{1}{r}\left[  1-\left(  2\frac{d}{r}\ \cos\left(
\pi-\theta\right)  -\left\{  \frac{d}{r}\right\}  ^{2}\right)  \right]
^{-1/2}%
\end{align*}
y consecuentemente
\begin{align*}
\frac{1}{R^{\prime}}  & =\frac{1}{r}\sum_{n=0}^{\infty}P_{n}(\cos
\theta)\ \left\{  \frac{d}{r}\right\}  ^{n}\\
\frac{1}{R}  & =\frac{1}{r}\sum_{n=0}^{\infty}P_{n}\left(  \cos\left(
\pi-\theta\right)  \right)  \ \left\{  \frac{d}{r}\right\}  ^{n}=\frac{1}%
{r}\sum_{n=0}^{\infty}P_{n}(-\cos\theta)\ \left\{  \frac{d}{r}\right\}  ^{n}%
\end{align*}
El potencial ser\'{a}
\[
V=\frac{q}{r}\left(  \sum_{n=0}^{\infty}\left[  P_{n}(\cos\theta)-P_{n}%
(-\cos\theta)\ \right]  \ \left\{  \frac{d}{r}\right\}  ^{n}\right)
\]
donde todos los t\'{e}rminos pares de $P_{n}(\cos\theta)$ se anula y
finalmente tendremos la expresi\'{o}n del potencial para cualquier punto del
plano
\[
V=\frac{2q}{r}\left(  \sum_{n=0}^{\infty}P_{2n+1}(\cos\theta)\ \left\{
\frac{d}{r}\right\}  ^{2n+1}\right)
\]
Entonces nos quedamos con el primer t\'{e}rmino de la serie, si
\[
\frac{d}{r}\ll1\quad\Rightarrow\quad V\approx\frac{q}{r^{2}}\ 2d\cos\theta
\]

\textbf{Series de Legendre}\newline Cualquier funci\'{o}n en el intervalo
$\left[  -1,1\right]  $ puede ser expresada en esa base.
\[
f(x)=|\mathbf{F}\rangle=\sum_{k=0}^{\infty}a_{k}\ |\mathbf{P}_{k}\rangle
=\sum_{k=0}^{\infty}\frac{\langle\mathbf{P}_{k}|\mathbf{F}\rangle}%
{\langle\mathbf{P}_{k}|\mathbf{P}_{k}\rangle}\ |\mathbf{P}_{k}\rangle
\]
Varios ejemplos ilustrar\'{a}n esta aplicaci\'{o}n

Si $f(x)$ es un polinomio
\[
f(x)=\sum_{n=0}^{m}b_{n}x^{n}=\sum_{k=0}^{\infty}a_{k}\ |\mathbf{P}_{k}%
\rangle=\sum_{n=0}^{\infty}a_{n}P_{n}(x)
\]
no se requiere hacer ninguna integral por cuanto los coeficientes $a_{n}$ se
determinan a trav\'{e}s de un sistema de ecuaciones algebraicas. Para el caso
de $f(x)=x^{2}$ tendremos
\begin{align*}
f(x)  & =x^{2}=a_{0}P_{0}(x)+a_{1}P_{1}(x)+a_{2}P_{2}(x)\\
f(x)  & =x^{2}=a_{0}+a_{1}x+\frac12a_{2}(3x^{2}-1)\\
f(x)  & =x^{2}=\frac13P_{0}(x)+\frac23P_{2}(x)
\end{align*}
Si
\[
f(x)=\sqrt{\frac{1-x}2}=\sum_{k=0}^{\infty}\frac{\langle\mathbf{P}%
_{k}|\mathbf{F}\rangle}{\langle\mathbf{P}_{k}|\mathbf{P}_{k}\rangle
}\ |\mathbf{P}_{k}\rangle
\]
\[
\langle\mathbf{P}_{k}|\mathbf{F}\rangle=\int_{-1}^{1}f(x)P_{k}(x)\mathrm{d}%
x=\int_{-1}^{1}\sqrt{\frac{1-x}2}P_{k}(x)\mathrm{d}x
\]
Sin embargo aqu\'{\i} muestra su utilidad la funci\'{o}n generatriz e
integrando. As\'{\i}
\begin{align*}%
%TCIMACRO{\dint _{-1}^{1}}%
%BeginExpansion
{\displaystyle\int_{-1}^{1}}
%EndExpansion
\sqrt{\frac{1-x}2}\left[  \frac1{\sqrt{1-2xt+t^{2}}}\right]  \mathrm{d}x  &
=\sum_{n=0}^{\infty}t^{n}%
%TCIMACRO{\dint _{-1}^{1}}%
%BeginExpansion
{\displaystyle\int_{-1}^{1}}
%EndExpansion
\sqrt{\frac{1-x}2}P_{n}(x)\mathrm{d}x\\
\frac1{2t}\left[  1+t-\frac{\left(  1-t\right)  ^{2}}{2\sqrt{t}}\ln\left(
\frac{1+\sqrt{t}}{1-\sqrt{t}}\right)  \right]   & =\sum_{n=0}^{\infty}t^{n}%
%TCIMACRO{\dint _{-1}^{1}}%
%BeginExpansion
{\displaystyle\int_{-1}^{1}}
%EndExpansion
\sqrt{\frac{1-x}2}P_{n}(x)\mathrm{d}x
\end{align*}
Expandiendo el lado izquierdo en series de potencias de $t$%
\[
\frac43-4\sum_{n=1}^{\infty}\frac{t^{n}}{\left(  4n^{2}-1\right)  \left(
2n+3\right)  }=\sum_{n=0}^{\infty}t^{n}%
%TCIMACRO{\dint _{-1}^{1}}%
%BeginExpansion
{\displaystyle\int_{-1}^{1}}
%EndExpansion
\sqrt{\frac{1-x}2}P_{n}(x)\mathrm{d}x
\]
lo cual nos conduce, al igualar coeficientes a
\begin{align*}
\frac43  & =%
%TCIMACRO{\dint _{-1}^{1}}%
%BeginExpansion
{\displaystyle\int_{-1}^{1}}
%EndExpansion
\sqrt{\frac{1-x}2}P_{0}(x)\mathrm{d}x\\
\frac{-4}{\left(  4n^{2}-1\right)  \left(  2n+3\right)  }  & =%
%TCIMACRO{\dint _{-1}^{1}}%
%BeginExpansion
{\displaystyle\int_{-1}^{1}}
%EndExpansion
\sqrt{\frac{1-x}2}P_{n}(x)\mathrm{d}x
\end{align*}
y finalmente a la forma de la expansi\'{o}n en series
\[
\sqrt{\frac{1-x}2}=\frac23P_{0}(x)-2\sum_{n=1}^{\infty}\frac{P_{n}(x)}{\left(
2n-1\right)  \left(  2n+3\right)  }
\]

\begin{center}
\textbf{Polinomios de Hermite}
\end{center}

Tal y como los polinomios de Legendre, los polinomios de Hermite, surgen como
soluciones particulares de una ecuaci\'{o}n diferencial
\[
y^{\prime\prime}-2xy^{\prime}+2\lambda y=0,
\]
para la cual todos los $x$ son ordinarios con $x\in(-\infty,\infty).$ Se
propone como soluci\'{o}n
\[
y(x)=\sum_{n=0}^{\infty}a_{n}x^{n}
\]
y se procede de la forma est\'{a}ndar. Tambi\'{e}n, al igual que los otros
polinomios ortogonales puede ser definido a partir de una ecuaci\'{o}n:
\begin{equation}
H_{\lambda}(x)=(-1)^{\lambda}e^{x^{2}}\frac{\mathrm{d}^{\lambda}}%
{\mathrm{d}x^{\lambda}}e^{-x^{2}},\qquad\lambda=0,1,2,....\label{HermiteDef}%
\end{equation}
obteniendo
\begin{align*}
H_{0}(x)  & =1;\qquad H_{1}(x)=2x;\qquad H_{2}(x)=4x^{2}-2;\qquad\\
H_{3}(x)  & =8x^{3}-12x\qquad H_{4}(x)=16x^{5}-48x^{2}+12\\
H_{5}(x)  & =32x^{5}-160x^{3}+120x\qquad
\end{align*}
y en general
\[
H_{\lambda}(x)=\sum_{k=0}^{\lambda/2}\frac{(-1)^{k}\lambda!}{k!\left(
\lambda-2k\right)  !}\left(  2x\right)  ^{\lambda-2k}
\]
\textbf{Funci\'{o}n Generatriz de los Polinomios de Hermite}\newline Se puede
encontrar una funci\'{o}n generatriz $\mathcal{H}(t,x)$ de los polinomios de
Hermite:
\[
\mathcal{H}(t,x)=e^{2xt-t^{2}}=H_{0}(x)\ +H_{1}(x)\ t+\frac{H_{2}(x)}%
2\ t^{2}+\frac{H_{3}(x)}{3!}\ t^{2}+\cdots=\sum_{n=0}^{\infty}\frac{H_{n}%
(x)}{n!}\ t^{n}
\]
para la cual los $H_{n}(x)$ son los coeficientes de su desarrollo en series de
potencias. Es f\'{a}cil darse cuenta que esta expresi\'{o}n proviene del
desarrollo en Serie de Taylor
\[
\mathcal{H}(t,x)=e^{2xt-t^{2}}=\sum_{n=0}^{\infty}\frac1{n!}\left[
\frac{\partial^{n}\mathcal{H}(t,x)}{\partial t^{n}}\right]  _{t=0}%
\ t^{n}\qquad\left\|  t\right\|  <\infty
\]
para lo cual
\[
\left[  \frac{\partial^{n}\mathcal{H}(t,x)}{\partial t^{n}}\right]
_{t=0}=e^{x^{2}}\left[  \frac{\partial^{n}}{\partial t^{n}}e^{-\left(
x-t\right)  ^{2}}\right]  _{t=0}=\left(  -1\right)  ^{n}e^{x^{2}}\left[
\frac{\mathrm{d}^{n}}{\mathrm{d}u^{n}}e^{-\left(  u\right)  ^{2}}\right]
_{u=x}=H_{n}(x)
\]

\textbf{Relaci\'{o}n de Recurrencia}\newline A partir de la funci\'{o}n
generatriz se puede construir la siguiente identidad
\[
\frac{\partial\mathcal{H}(t,x)}{\partial t}=\left(  2x-2t\right)  \mathcal{H}
\]
y utilizando el desarrollo en series de potencias en $t$ tendremos,
\begin{align*}
\sum_{n=1}^{\infty}\frac{H_{n}(x)}{n!}\ nt^{n-1}-2x\sum_{n=0}^{\infty}%
\frac{H_{n}(x)}{n!}\ t^{n}+\sum_{n=0}^{\infty}\frac{H_{n}(x)}{n!}\ t^{n+1}  &
=0\\
\sum_{n=0}^{\infty}\ \frac1{n!}\left[  \underbrace{H_{n+1}(x)-2xH_{n}%
(x)+2nH_{n-1}(x)}_{=0}\right]  t^{n}  & =0
\end{align*}
As\'{\i} la relaci\'{o}n de recurrencia ser\'{a}
\[
H_{n+1}(x)-2xH_{n}(x)+2nH_{n-1}(x)=0
\]
De igual modo, podemos partir de otra identidad
\[
\frac{\partial\mathcal{H}(t,x)}{\partial x}=2t\ \mathcal{H}\Rightarrow
\sum_{n=0}^{\infty}\frac{H_{n}^{\prime}(x)}{n!}\ t^{n}-2\sum_{n=0}^{\infty
}\frac{H_{n}(x)}{n!}\ t^{n+1}
\]
y encontrar una relaci\'{o}n para generar las derivadas de los polinomios de
Hermite en t\'{e}rmino de ellos mismos:
\[
H_{n}^{\prime}(x)=2n\ H_{n-1}(x),\qquad n=1,2,3,\cdots
\]
Finalmente, utilizando la ecuaci\'{o}n anterior en la relaci\'{o}n de
recurrencia y derivando esa expresi\'{o}n una vez m\'{a}s, queda como:
\begin{align*}
H_{n+1}(x)-2xH_{n}(x)+H_{n}^{\prime}(x)  & =0\\
H_{n}^{\prime\prime}(x)-2xH_{n}^{\prime}(x)+2n\ H_{n}(x)  & =0
\end{align*}
con lo cual queda demostrado que los polinomios de Hermite son una
soluci\'{o}n particular de esa ecuaci\'{o}n diferencial.
\[
y^{\prime\prime}-2xy^{\prime}+2ny=0,
\]
Donde hemos hecho $y=H_{n}(x)$ Adicionalmente, haciendo un cambio
cosm\'{e}tico podremos demostrar que $y=e^{-x^{2}/2}H_{n}(x)$ es soluci\'{o}n
de la ecuaci\'{o}n diferencial autoadjunta
\[
y^{\prime\prime}+\left(  2n+1-x^{2}\right)  y=0
\]

\textbf{Ortogonalidad y Norma de los Polinomios de Hermite}\newline En general
estos polinomios cumplen con
\[
\langle\mathbf{H}_{\alpha}|\mathbf{H}_{\beta}\rangle=2^{n}n!\sqrt{\pi}%
\ \delta_{\alpha\beta}=\int_{-\infty}^{\infty}e^{-x^{2}}H_{\beta}(x)H_{\alpha
}(x)\mathrm{d}x
\]
Donde la funci\'{o}n delta de Kronecker es $\delta_{\alpha\beta}=0$ si
$\alpha\neq\beta$; y $\delta_{\beta\beta}=1.$

Para demostrar el caso $\alpha\neq\beta$ partimos de
\begin{align*}
u_{\beta}\left[  u_{\alpha}^{\prime\prime}+\left(  2\alpha+1-x^{2}\right)
u_{\alpha}\right]   & =0\\
u_{\alpha}\left[  u_{\beta}^{\prime\prime}+\left(  2\beta+1-x^{2}\right)
u_{\beta}\right]   & =0
\end{align*}
restando miembro a miembro e integrando se tiene que:
\begin{align*}
\left[  u_{\alpha}^{\prime}u_{\beta}-u_{\beta}^{\prime}u_{\alpha}\right]
^{\prime}+2\left(  \alpha-\beta\right)  u_{\alpha}u_{\beta} & =0\\
\left(  \alpha-\beta\right)  \int_{-\infty}^{\infty}e^{-x^{2}}H_{\beta
}(x)H_{\alpha}(x)\mathrm{d}x  & =0\\
\int_{-\infty}^{\infty}e^{-x^{2}}H_{\beta}(x)H_{\alpha}(x)\mathrm{d}x  &
=0\qquad\alpha\neq\beta;
\end{align*}
ya que
\[
\left.  e^{-x^{2}/2}\left(  2\alpha\ H_{\alpha-1}(x)H_{\beta}(x)-2\beta
\ H_{\beta-1}(x)H_{\alpha}(x)\right)  \right|  _{-\infty}^{\infty}=0
\]
Para encontrar el valor de la norma, procedemos a partir de la relaci\'{o}n de
recurrencia
\begin{align*}
H_{n}(x)\left(  H_{n}(x)-2xH_{n-1}(x)+2(n-1)H_{n-2}(x)\right)   & =0\\
H_{n-1}(x)\left(  H_{n+1}(x)-2xH_{n}(x)+2nH_{n-1}(x)\right)   & =0
\end{align*}
restando miembro a miembro, multiplicando por $e^{-x^{2}}$ e integrando entre
$(-\infty,\infty)$ se obtiene
\[
\int_{-\infty}^{\infty}e^{-x^{2}}H_{\alpha}^{2}(x)\mathrm{d}x=2\alpha
\int_{-\infty}^{\infty}e^{-x^{2}}H_{\alpha-1}^{2}(x)\mathrm{d}x
\]
repitiendo la operaci\'{o}n y recordando que al final queda
\[
\int_{-\infty}^{\infty}e^{-x^{2}}x^{2}\mathrm{d}x=2\sqrt{\pi}
\]
Obtenemos
\[
\langle\mathbf{H}_{\alpha}|\mathbf{H}_{\alpha}\rangle=\left\|  \mathbf{H}%
_{\alpha}\right\|  ^{2}=\int_{-\infty}^{\infty}e^{-x^{2}}H_{\alpha}%
^{2}(x)\mathrm{d}x=2^{n}n!\sqrt{\pi}
\]

\textbf{Representaci\'{o}n Integral de los Polinomios de Hermite}\newline Los
polinomios de Hermite pueden ser representados como
\[
H_{n}(x)=\frac{2^{n}(-i)^{n}e^{x^{2}}}{\sqrt{\pi}}\int_{-\infty}^{\infty
}e^{-t^{2}+2itx}t^{n}\mathrm{d}t
\]
que puede ser separada como
\[
H_{2n}(x)=\frac{2^{2n+1}(-1)^{n}e^{x^{2}}}{\sqrt{\pi}}\int_{0}^{\infty
}e^{-t^{2}}t^{2n}\cos2xt\ \mathrm{d}t\qquad n=1,2,3,\cdots
\]
y paralos t\'{e}rminos impares
\[
H_{2n+1}(x)=\frac{2^{2n+2}(-1)^{n}e^{x^{2}}}{\sqrt{\pi}}\int_{0}^{\infty
}e^{-t^{2}}t^{2n+1}\operatorname*{sen}2xt\ \mathrm{d}t\qquad n=1,2,3,\cdots
\]
La forma de llegar a cualquiera de estas \'{u}ltimas f\'{o}rmulas se parte de
las conocidas integrales desarrolladas en el plano complejo
\[
e^{-x^{2}}=\frac2{\sqrt{\pi}}\int_{-\infty}^{\infty}e^{-t^{2}}\cos
2xt\ \mathrm{d}t
\]
se deriva $2n$ veces a ambos miembros se utiliza la definici\'{o}n de los
polinomios de Hermite.

\textbf{Series de Hermite}\newline Antes de desarrollar funciones en
t\'{e}rminos de los polinomios de Hermite, expondremos un par de teoremas sin
demostraci\'{o}n.

\textbf{Teorema 1}\newline Sean $|\ \mathbf{f\ }\rangle$ y $|\ \mathbf{g\ }%
\rangle$ dos funciones arbitrarias, cuando menos continuas a trozos en
$\left(  -\infty,\infty\right)  $ y que cumplen con
\[
\int_{-\infty}^{\infty}e^{-x^{2}}f^{2}(x)\mathrm{d}x<\infty\qquad\wedge
\qquad\int_{-\infty}^{\infty}e^{-x^{2}}g^{2}(x)\mathrm{d}x<\infty
\]
Entonces el conjunto de estas funciones forman un espacio vectorial Euclideano
$\mathcal{I}_{2}^{w}$ con un producto interno definido por
\[
\langle\ \mathbf{g}\ |\ \mathbf{f\ }\rangle=\int_{-\infty}^{\infty}e^{-x^{2}%
}f(x)g(x)\mathrm{d}x
\]
Las funciones $f(x)$ y $g(x)$ se denominan cuadrado-integrables respecto al
peso $w$. Es por ello que denotamos el espacio de funciones como
$\mathcal{I}_{2}^{w}$

\textbf{Teorema 2}\newline Si $f(x)$ es una funci\'{o}n continua arbitraria en
$\mathcal{I}_{2}^{w}$ entonces puede ser aproximada por un polinomio en ese
mismo espacio. Es decir
\[
\lim_{n\rightarrow\infty}\left\|  f(x)-p_{n}(x)\right\|  =\lim_{n\rightarrow
\infty}\left(  \int_{-\infty}^{\infty}e^{-x^{2}}\left[  f(x)-p_{n}(x)\right]
^{2}\mathrm{d}x\right)  ^{1/2}=0
\]
As\'{\i}, la expresi\'{o}n de una funci\'{o}n arbitraria en la base de los
polinomio de Hermite se reduce a
\[
f(x)=|\ \mathbf{f\ }\rangle=\sum_{k=0}^{\infty}a_{k}\ |\mathbf{H}_{k}%
\rangle=\sum_{k=0}^{\infty}\frac{\langle\mathbf{H}_{k}|\mathbf{f}\rangle
}{\langle\mathbf{H}_{k}|\mathbf{H}_{k}\rangle}\ |\mathbf{H}_{k}\rangle
\]
donde
\[
a_{k}=\frac{\langle\mathbf{H}_{k}|\ \mathbf{f\ }\rangle}{\langle\mathbf{H}%
_{k}|\mathbf{H}_{k}\rangle}\ =\frac{\int_{-\infty}^{\infty}e^{-x^{2}}%
f(x)H_{k}(x)\mathrm{d}x}{\int_{-\infty}^{\infty}e^{-x^{2}}H_{k}^{2}%
(x)\mathrm{d}x}=\frac1{2^{k}k!\sqrt{\pi}}\int_{-\infty}^{\infty}e^{-x^{2}%
}f(x)H_{k}(x)\mathrm{d}x
\]
Si $f(x)=x^{2p}$ con $p=1,2,3,\cdots$
\[
f(x)=x^{2p}=\sum_{k=0}^{p}a_{2k}\ H_{2k}(x)
\]
entonces
\begin{align}
a_{2k}  & =\frac1{2^{2k}(2k)!\sqrt{\pi}}\int_{-\infty}^{\infty}e^{-x^{2}%
}x^{2p}H_{2k}(x)\mathrm{d}x\\[0.03in]
& =\frac1{2^{2k}(2k)!\sqrt{\pi}}\int_{-\infty}^{\infty}x^{2p}\frac
{\mathrm{d}^{2k}}{\mathrm{d}x^{2k}}e^{-x^{2}}\mathrm{d}x\nonumber
\end{align}
Una integraci\'{o}n por partes estrat\'{e}gica muestra que:
\[
a_{2k}=\frac1{2^{2k}(2k)!\sqrt{\pi}}\left\{  \left.  x^{2p}\frac
{\mathrm{d}^{2k-1}}{\mathrm{d}x^{2k-1}}e^{-x^{2}}\right|  _{-\infty}^{\infty
}-\int_{-\infty}^{\infty}2px^{2p-1}\frac{\mathrm{d}^{2k-1}}{\mathrm{d}%
x^{2k-1}}e^{-x^{2}}\mathrm{d}x\right\}
\]
El primer t\'{e}rmico de la resta se anula siempre debido a la defici\'{o}n de
los polinomios de Hermite
\[
\left.  x^{2p}\frac{\mathrm{d}^{2k-1}}{\mathrm{d}x^{2k-1}}e^{-x^{2}}\right|
_{-\infty}^{\infty}=\left.  x^{2p}(-1)^{2k-1}e^{-x^{2}}H_{2k-1}(x)\right|
_{-\infty}^{\infty}%
\]
Repitiendo el proceso $2k$ veces, tendremos
\[
a_{2k}=\frac1{2^{2k}(2k)!\sqrt{\pi}}\frac{\left(  2p\right)  !}{\left(
2p-2k\right)  !}\int_{-\infty}^{\infty}x^{2p-2k}\ e^{-x^{2}}\mathrm{d}x
\]
ahora si en la integralhacemos $x=\sqrt{t}$ obtenemos
\begin{align*}
a_{2k}  & =\frac1{2^{2k}(2k)!\sqrt{\pi}}\frac{\left(  2p\right)  !}{\left(
2p-2k\right)  !}\int_{-\infty}^{\infty}t^{p-k}\ e^{-t}\frac{\mathrm{d}%
t}{2\sqrt{t}}\\
& =\frac1{2^{2k+1}(2k)!\sqrt{\pi}}\frac{\left(  2p\right)  !}{\left(
2p-2k\right)  !}\int_{-\infty}^{\infty}t^{p-k-\frac12}\ e^{-t}\mathrm{d}t
\end{align*}
y utilizando la definici\'{o}n $\Gamma\left(  z\right)  \equiv\int_{0}%
^{\infty}e^{-t}t^{z-1}\mathrm{d}t\equiv\left(  z-1\right)  !$ , queda como
\[
a_{2k}=\frac1{2^{2k+1}(2k)!\sqrt{\pi}}\frac{\left(  2p\right)  !}{\left(
2p-2k\right)  !}\Gamma\left(  p-k+\frac12\right)
\]
Ahora, recurrimos a la propiedad de ``duplicaci\'{o}n'' de la Funci\'{o}n
Gamma, i.e.
\[
2^{2z-1}\Gamma\left(  z\right)  \Gamma\left(  z+\frac12\right)  =\sqrt{\pi
}\Gamma\left(  2z\right)
\]
tenemos que
\[
2^{2p-2k}\Gamma\left(  p-k+\frac12\right)  \left(  p-k\right)  !=\sqrt{\pi
}\left(  2p-2k\right)  !
\]
quedan entonces los coeficientes determinados como
\[
a_{2k}=\frac{\left(  2p\right)  !}{2^{2p+1}(2k)!\left(  p-k\right)  !}
\]
y, por lo tanto el desarrollo en la base de los polinomios de Hermite
\[
f(x)=x^{2p}=\frac{\left(  2p\right)  !}{2^{2p+1}}\sum_{k=0}^{p}\ \frac
{H_{2k}(x)}{(2k)!\left(  p-k\right)  !}\qquad-\infty<x<\infty
\]
Muestre que del mismo modo se puede encontrar
\[
f(x)=x^{2p+1}=\frac{\left(  2p-1\right)  !}{2^{2p-1}}\sum_{k=0}^{p}%
\ \frac{H_{2k+1}(x)}{(2k+1)!\left(  p-k\right)  !}\qquad-\infty<x<\infty
\]
Si $f(x)=e^{-a^{2}x^{2}}$ con $\operatorname{Re}a^{2}>-1.$ Otra vez
\[
f(x)=e^{-a^{2}x^{2}}=\sum_{k=0}^{\infty}a_{2k}\ H_{2k}(x)
\]
entonces
\begin{equation}
a_{2k}=\frac1{2^{2k}(2k)!\sqrt{\pi}}\int_{-\infty}^{\infty}e^{-(a^{2}+1)x^{2}%
}H_{2k}(x)\mathrm{d}x\nonumber
\end{equation}
Sustituyendo $H_{2k}(x)$ por su expresi\'{o}n integral tendremos
\begin{align*}
a_{2k}  & =\frac1{2^{2k}(2k)!\sqrt{\pi}}\int_{-\infty}^{\infty}e^{-(a^{2}%
+1)x^{2}}\left[  \frac{2^{2k+1}(-1)^{k}e^{x^{2}}}{\sqrt{\pi}}\int_{0}^{\infty
}e^{-t^{2}}t^{2k}\cos2xt\ \mathrm{d}t\right]  \mathrm{d}x\\
& =\frac{2(-1)^{k}}{\pi(2k)!}\int_{-\infty}^{\infty}e^{-a^{2}x^{2}}\left[
\int_{0}^{\infty}e^{-t^{2}}t^{2k}\cos2xt\ \mathrm{d}t\right]  \mathrm{d}x\\
& \equiv\frac{2(-1)^{k}}{\pi(2k)!}\int_{0}^{\infty}e^{-t^{2}}t^{2k}\left[
\int_{-\infty}^{\infty}e^{-a^{2}x^{2}}\cos2xt\ \mathrm{d}x\right]
\ \mathrm{d}t\\
& =\frac{2(-1)^{k}}{\pi(2k)!}\int_{0}^{\infty}e^{-t^{2}}t^{2k}\left[
\sqrt{\frac\pi{a^{2}}}\ e^{-t^{2}/a^{2}}\right]  \ \mathrm{d}t=\\
& =\frac{2(-1)^{k}}{\sqrt{\pi}(2k)!a}\int_{0}^{\infty}e^{-t^{2}(1+a^{-2}%
)}\ t^{2k}\ \mathrm{d}t\\
& =\frac{(-1)^{k}}{\sqrt{\pi}(2k)!}\ \frac{a^{2k}}{\left(  1+a^{2}\right)
^{k+1/2}}\int_{0}^{\infty}e^{-s}\ s^{k-\frac12}\ \mathrm{d}s\qquad\leftarrow
t^{2}(1+a^{-2})=s\\
& =\frac{(-1)^{k}}{\sqrt{\pi}(2k)!}\ \frac{a^{2k}}{\left(  1+a^{2}\right)
^{k+1/2}}\Gamma\left(  k+\frac12\right)
\end{align*}
y ahora usando, otra vez la propiedad de ``duplicaci\'{o}n'' de la funci\'{o}n
gamma,
\[
2^{2k}\Gamma\left(  k+\frac12\right)  k!=\sqrt{\pi}\left(  2k\right)  !
\]
obtenemos
\[
a_{2k}=\frac{(-1)^{k}a^{2k}}{2^{2k}\ k!\left(  1+a^{2}\right)  ^{k+1/2}}\
\]
por lo tanto
\[
f(x)=e^{-a^{2}x^{2}}=\sum_{k=0}^{\infty}\frac{(-1)^{k}a^{2k}}{2^{2k}%
\ k!\left(  1+a^{2}\right)  ^{k+1/2}}\ H_{2k}(x)
\]

\begin{center}
\textbf{Un Ejemplo de la F\'{\i}sica:}

\textit{El Oscilador arm\'{o}nico, independiente del Tiempo, en Mec\'{a}nica
Cu\'{a}ntica.}
\end{center}

La Ecuaci\'{o}n de Schr\"{o}dinger independiente del tiempo y en una
dimensi\'{o}n es
\[
\frac{\mathrm{d}^{2}}{\mathrm{d}x^{2}}\psi(x)+\frac{2\mu}{\hbar^{2}}\left[
E-\mathcal{U}(x)\right]  \psi(x)=0
\]
con $\mu$ la ``masa'' de la part\'{i}cula;$\ E$ los niveles de energ\'{i}a y
$\mathcal{U}(x)$ el potencial al cual est\'{a} sometida la part\'{i}cula. En
el caso que estudiemos un potencial $\mathcal{U}(x)=\frac{1}{2}\mu\omega
^{2}x^{2}$ en el cual la frecuencia angular del oscilador viene representada
por $\omega$. La ecuaci\'{o}n de Schr\"{o}dinger se convierte en
\[
\frac{\mathrm{d}^{2}}{\mathrm{d}x^{2}}\psi(x)+\frac{2\mu}{\hbar^{2}}\left[
E-\frac{1}{2}\mu\omega^{2}x^{2}\right]  \psi(x)=0
\]
haciendo un cambio de variable $\xi=x\sqrt{\mu\omega/\hbar}$ para
adimensionalizar la ecuaci\'{o}n, se obtiene
\[
\psi^{\prime\prime}(\xi)+\left[  \frac{2E}{\hbar\omega}-\xi^{2}\right]
\psi(\xi)=0
\]
la cual corresponde a la forma autoadjunta de la Ecuaci\'{o}n de Hermite y por
lo tanto identificamos
\[
\frac{2E}{\hbar\omega}=2n+1\quad\Rightarrow\quad E=\left(  n+\frac{1}%
{2}\right)  \hbar\omega
\]
con lo cual comprobamos la forma como viene cuantizada la energ\'{i}a en este
sistema y la energ\'{i}a del estado fundamental. Por su parte, la funci\'{o}n
de onda se podr\'{a} expresar en la base de soluciones de esa ecuaci\'{o}n
\[
\psi(\xi)=\sum_{n=0}^{\infty}c_{n}\ \psi_{n}(\xi)=\sum_{n=0}^{\infty}%
c_{n}\ e^{-\xi^{2}/2}H_{n}(\xi)
\]

\begin{center}%
%TCIMACRO{\FRAME{ftbpF}{4.5238in}{3.3987in}{0pt}{}{}{hermite1.jpg}%
%{\special{ language "Scientific Word";  type "GRAPHIC";
%maintain-aspect-ratio TRUE;  display "USEDEF";  valid_file "F";
%width 4.5238in;  height 3.3987in;  depth 0pt;  original-width 9.9895in;
%original-height 7.4893in;  cropleft "0";  croptop "1";  cropright "1";
%cropbottom "0";  filename 'hermite1.JPG';file-properties "XNPEU";}}}%
%BeginExpansion
\begin{figure}
[ptb]
\begin{center}
\includegraphics[
natheight=7.489300in,
natwidth=9.989500in,
height=3.3987in,
width=4.5238in
]%
{hermite1.jpg}%
\end{center}
\end{figure}
%EndExpansion
\end{center}

y se mantenemos la normalizaci\'{o}n
\[
\int_{-\infty}^{\infty}\psi_{n}^{2}(\xi)\mathrm{d}\xi=1
\]
podremos expresar los coeficientes como
\[
c_{n}=\left(  \frac{\mu\omega}{\pi\hbar}\right)  ^{1/4}\frac{1}{\sqrt{2^{n}%
n!}}%
\]
\ 

\begin{center}
\textbf{Resumen de Propiedades Polinomios Ortogonales}%

\begin{tabular}
[c]{|l|l|}\hline
\multicolumn{2}{|c|}{\textbf{Polinomios de Legendre}}\\\hline
Definici\'{o}n & $P_{n}(x)=\dfrac1{n!2^{n}}\dfrac{\mathrm{d}^{n}}%
{\mathrm{d}x^{n}}(x^{2}-1)^{n},\qquad n=0,1,2,.....$\\\hline
Ejemplos & $%
\begin{array}
[c]{c}%
P_{-1}\equiv0;\quad P_{0}\equiv1;\quad P_{1}=x\\
P_{2}=\frac12(3x^{2}-1);\quad P_{3}=\frac12(5x^{3}-3x)
\end{array}
$\\\hline
Relaci\'{o}n de Recurrencia & $\left(  n+1\right)  P_{n+1}(x)=\ \left(
2n+1\right)  xP_{n}(x)-nP_{n-1}(x)$\\\hline
Ecuaciones Diferenciales & $%
\begin{array}
[c]{c}%
(1-x^{2})\ y^{\prime\prime}-2x\ y^{\prime}+\lambda(\lambda+1)\ y=0\\
\dfrac1{\operatorname*{sen}\theta}\dfrac{\mathrm{d}}{\mathrm{d}\theta}\left(
\operatorname*{sen}\theta\ \dfrac{\mathrm{d}u}{\mathrm{d}\theta}\right)
+n(n+1)u=0;\quad u=P_{n}(\cos\theta)
\end{array}
$\\\hline
Funci\'{o}n Generatriz & $\mathcal{P}(t,x)=\dfrac1{\sqrt{1-2xt+t^{2}}}=%
%TCIMACRO{\dsum _{n=0}^{\infty}}%
%BeginExpansion
{\displaystyle\sum_{n=0}^{\infty}}
%EndExpansion
P_{n}(x)\ t^{n}$\\\hline
Representaci\'{o}n Integral & $P_{n}(x)=\dfrac1{2\pi}%
%TCIMACRO{\dint _{0}^{\pi}}%
%BeginExpansion
{\displaystyle\int_{0}^{\pi}}
%EndExpansion
\left[  x+\sqrt{x^{2}-1}\cos\varphi\right]  ^{n}\mathrm{d}\varphi$\\\hline
Ortogonalidad & $\langle\mathbf{P}_{\alpha}|\mathbf{P}_{\beta}\rangle=%
%TCIMACRO{\dint _{-1}^{1}}%
%BeginExpansion
{\displaystyle\int_{-1}^{1}}
%EndExpansion
P_{\alpha}(x)P_{\beta}(x)\mathrm{d}x=\delta_{\alpha\beta}\dfrac2{2\alpha+1}%
$\\\hline
\end{tabular}%

\begin{tabular}
[c]{|l|l|}\hline
\multicolumn{2}{|c|}{\textbf{Polinomios de Hermite}}\\\hline
Definici\'{o}n & $%
\begin{array}
[c]{c}%
H_{n}(x)=(-1)^{n}e^{x^{2}}\dfrac{\mathrm{d}^{n}}{\mathrm{d}x^{n}}e^{-x^{2}%
},\qquad n=0,1,2,....\\
H_{n}(x)=%
%TCIMACRO{\dsum _{k=0}^{n/2}}%
%BeginExpansion
{\displaystyle\sum_{k=0}^{n/2}}
%EndExpansion
\dfrac{(-1)^{k}n!}{k!\left(  n-2k\right)  !}\left(  2x\right)  ^{n-2k}%
\qquad\qquad
\end{array}
$\\\hline
Ejemplos & $%
\begin{array}
[c]{c}%
H_{0}(x)=1;\qquad H_{1}(x)=2x;\qquad H_{2}(x)=4x^{2}-2;\\
H_{3}(x)=8x^{3}-12x\qquad H_{4}(x)=16x^{5}-48x^{2}+12
\end{array}
$\\\hline
Relaciones de Recurrencia & $%
\begin{array}
[c]{c}%
H_{n+1}(x)-2xH_{n}(x)+2nH_{n-1}(x)=0\\
H_{n}^{\prime}(x)=2n\ H_{n-1}(x),\qquad n=1,2,3,\cdots
\end{array}
$\\\hline
Ecuaciones Diferenciales & $%
\begin{array}
[c]{c}%
y^{\prime\prime}-2xy^{\prime}+2ny=0\\
u^{\prime\prime}+\left(  2n+1-x^{2}\right)  u=0;\quad u(x)=e^{-x^{2}/2}%
H_{n}(x)
\end{array}
$\\\hline
Funci\'{o}n Generatriz & $\mathcal{H}(t,x)=\mathbf{e}^{2xt-t^{2}}=%
%TCIMACRO{\dsum _{n=0}^{\infty}}%
%BeginExpansion
{\displaystyle\sum_{n=0}^{\infty}}
%EndExpansion
\dfrac{H_{n}(x)}{n!}\ t^{n}$\\\hline
Representaci\'{o}n Integral & $%
\begin{array}
[c]{c}%
H_{2n}(x)=\dfrac{2^{2n+1}(-1)^{n}e^{x^{2}}}{\sqrt{\pi}}%
%TCIMACRO{\dint _{0}^{\infty}}%
%BeginExpansion
{\displaystyle\int_{0}^{\infty}}
%EndExpansion
e^{-t^{2}}t^{2n}\cos2xt\ \mathrm{d}t\\
H_{2n+1}(x)=\dfrac{2^{2n+2}(-1)^{n}e^{x^{2}}}{\sqrt{\pi}}%
%TCIMACRO{\dint _{0}^{\infty}}%
%BeginExpansion
{\displaystyle\int_{0}^{\infty}}
%EndExpansion
e^{-t^{2}}t^{2n+1}\operatorname*{sen}2xt\ \mathrm{d}t
\end{array}
$\\\hline
Ortogonalidad & $\langle\mathbf{H}_{\alpha}|\mathbf{H}_{\beta}\rangle
=2^{n}n!\sqrt{\pi}\ \delta_{\alpha\beta}=%
%TCIMACRO{\dint _{-\infty}^{\infty}}%
%BeginExpansion
{\displaystyle\int_{-\infty}^{\infty}}
%EndExpansion
e^{-x^{2}}H_{\beta}(x)H_{\alpha}(x)\mathrm{d}x$\\\hline
\end{tabular}%

\begin{tabular}
[c]{|l|l|}\hline
\multicolumn{2}{|c|}{Polinomios de Laguerre Generalizados ($\alpha\neq0$%
)}\\\hline
Definici\'{o}n & $L_{n}^{\alpha}(x)=e^{x}\dfrac{x^{-\alpha}}{n!}%
\dfrac{\mathrm{d}^{n}}{\mathrm{d}x^{n}}(e^{-x}x^{n+\alpha}),\qquad
n=0,1,2,.....$\\\hline
Ejemplos & $%
\begin{array}
[c]{c}%
L_{0}^{\alpha}(x)\equiv1;\quad L_{1}^{\alpha}(x)=1+\alpha-x\\
L_{2}^{\alpha}(x)=\frac{1}{2}\left\{  \left(  1+\alpha\right)  \left(
2+\alpha\right)  -2\left(  2+\alpha\right)  x+x^{2}\right\}
\end{array}
$\\\hline
Relaciones de Recurrencia & $%
\begin{array}
[c]{c}%
\left(  n+1\right)  L_{n+1}^{\alpha}(x)+\left(  x-\alpha-2n-1\right)
L_{n}^{\alpha}(x)+\left(  n+\alpha\right)  L_{n-1}^{\alpha}(x)=0\\
x\dfrac{\mathrm{d}L_{n}^{\alpha}(x)}{\mathrm{d}x}=nL_{n}^{\alpha}(x)-\left(
n-\alpha\right)  L_{n-1}^{\alpha}(x)\\
L_{n}^{\alpha+1}(x)=L_{n}^{\alpha}(x)+L_{n-1}^{\alpha-1}(x)\\
\dfrac{\mathrm{d}L_{n}^{\alpha}(x)}{\mathrm{d}x}=-nL_{n-1}^{\alpha+1}(x)
\end{array}
$\\\hline
Ecuaciones Diferenciales & $%
\begin{array}
[c]{c}%
xy^{\prime\prime}+\left(  \alpha+-x\right)  y^{\prime}+ny=0\\
xu^{\prime\prime}+\left(  \alpha+1-2\nu\right)  u^{\prime}+\left(
n+\dfrac{\alpha+1}{2}-\dfrac{x}{4}+\dfrac{\nu\left(  \nu-\alpha\right)  }%
{x}\right)  u=0\\
u(x)=e^{-x^{2}/2}x^{\nu}L_{n}^{\alpha}(x)
\end{array}
$\\\hline
Funci\'{o}n Generatriz & $L(t,x)=\left(  1-t\right)  ^{-\alpha-1}%
e^{-xt/(1-t)}=%
%TCIMACRO{\dsum _{n=0}^{\infty}}%
%BeginExpansion
{\displaystyle\sum_{n=0}^{\infty}}
%EndExpansion
L_{n}^{\alpha}(x)\ t^{n}$\\\hline
Representaci\'{o}n Integral & $L_{n}^{\alpha}(x)=\dfrac{e^{x}x^{-\alpha/2}%
}{n!}%
%TCIMACRO{\dint _{0}^{\infty}}%
%BeginExpansion
{\displaystyle\int_{0}^{\infty}}
%EndExpansion
t^{\left(  n+\alpha/2\right)  }J_{\alpha}\left(  2\sqrt{xt}\right)
e^{-t}\ dt$\\\hline
Ortogonalidad & $\langle L_{n}^{\alpha}|L_{m}^{\alpha}\rangle=%
%TCIMACRO{\dint _{0}^{\infty}}%
%BeginExpansion
{\displaystyle\int_{0}^{\infty}}
%EndExpansion
e^{-x}x^{\alpha}\ L_{n}^{\alpha}(x)L_{m}^{\alpha}(x)\ dx=\delta_{nm}%
\dfrac{\Gamma\left(  n+\alpha+1\right)  }{n!}$\\\hline
\end{tabular}

\newpage\ \textbf{Ejemplos en la Expansi\'{o}n de Funciones en }

\textbf{T\'{e}rminos de Polinomios de Laguerre}
\end{center}

Otro Teorema sin demostraci\'{o}n

\textbf{Teorema 3}\newline Toda funci\'{o}n $f(x)$ continua a trozos, definida
en el intervalo infinito $\left(  0,\infty\right)  ,$ podr\'{a} ser
representada como
\[
f(x)=|\mathbf{F}\rangle=\sum_{n=0}^{\infty}c_{n}\ |\mathbf{L}_{n}^{\alpha
}\rangle=\sum_{n=0}^{\infty}\frac{\langle\mathbf{L}_{n}^{\alpha}%
|\mathbf{F}\rangle}{\langle\mathbf{L}_{n}^{\alpha}|\mathbf{L}_{n}^{\alpha
}\rangle}\ |\mathbf{L}_{n}^{\alpha}\rangle\equiv\sum_{n=0}^{\infty}%
c_{n}(x)L_{n}^{\alpha}(x)
\]
con
\[
c_{n}(x)=\frac{\langle\mathbf{L}_{n}^{\alpha}|\mathbf{F}\rangle}%
{\langle\mathbf{L}_{n}^{\alpha}|\mathbf{L}_{n}^{\alpha}\rangle}\equiv
\dfrac{n!}{\Gamma\left(  n+\alpha+1\right)  }%
%TCIMACRO{\dint _{0}^{\infty}}%
%BeginExpansion
{\displaystyle\int_{0}^{\infty}}
%EndExpansion
e^{-x}x^{\alpha}\ f(x)L_{n}^{\alpha}(x)\ \mathrm{d}x
\]
si
\[%
%TCIMACRO{\dint _{0}^{\infty}}%
%BeginExpansion
{\displaystyle\int_{0}^{\infty}}
%EndExpansion
e^{-x}x^{\alpha}\ f^{2}(x)\ \mathrm{d}x
\]
es finita.

Como un ejemplo del uso de este teorema, calcularemos la expansi\'{o}n de
$f(x)=x^{\nu},$ por lo tanto
\[
f(x)=x^{\nu}=\sum_{n=0}^{\infty}c_{n}(x)L_{n}^{\alpha}(x)
\]
con
\[
c_{n}(x)=\dfrac{n!}{\Gamma\left(  n+\alpha+1\right)  }%
%TCIMACRO{\dint _{0}^{\infty}}%
%BeginExpansion
{\displaystyle\int_{0}^{\infty}}
%EndExpansion
e^{-x}x^{\alpha+\upsilon}\ L_{n}^{\alpha}(x)\ \mathrm{d}x
\]
Sustituyendo la definici\'{o}n e integrando por partes
\begin{align*}
c_{n}(x)  & =\dfrac{1}{\Gamma\left(  n+\alpha+1\right)  }%
%TCIMACRO{\dint _{0}^{\infty}}%
%BeginExpansion
{\displaystyle\int_{0}^{\infty}}
%EndExpansion
x^{\upsilon}\ \dfrac{\mathrm{d}^{n}}{\mathrm{d}x^{n}}(e^{-x}x^{n+\alpha
})\ \mathrm{d}x\\
& =\dfrac{1}{\Gamma\left(  n+\alpha+1\right)  }\left\{  \left.  x^{\nu}%
\frac{\mathrm{d}^{n-1}}{\mathrm{d}x^{n-1}}\left(  e^{-x}x^{n+\alpha}\right)
\right|  _{0}^{\infty}-\int_{0}^{\infty}\nu x^{\nu-1}\frac{\mathrm{d}^{n-1}%
}{\mathrm{d}x^{n-1}}\left(  e^{-x}x^{n+\alpha}\right)  \mathrm{d}x\right\}
\end{align*}
El primer t\'{e}rmino de la resta se anula siempre debido a la definici\'{o}n
de los polinomios de Laguerre
\[
\left.  x^{\nu}\frac{\mathrm{d}^{n-1}}{\mathrm{d}x^{n-1}}\left(
e^{-x}x^{n+\alpha}\right)  \right|  _{0}^{\infty}=\left.  L_{n-1}^{\alpha
}(x)e^{-x}x^{\nu+\alpha}n!\right|  _{0}^{\infty}\equiv0
\]
Repitiendo el proceso $n$ veces, tendremos
\begin{align*}
c_{n}(x)  & =\dfrac{(-1)^{n}\nu(\nu-1)(\nu-2)\cdots(\nu-n+1)}{\Gamma\left(
n+\alpha+1\right)  }\int_{0}^{\infty}e^{-x}x^{\nu+\alpha}\mathrm{d}x\\
& =(-1)^{n}\dfrac{\Gamma\left(  \nu+\alpha+1\right)  \Gamma\left(
\nu+1\right)  }{\Gamma\left(  n+\alpha+1\right)  \Gamma\left(  \nu-n+1\right)
}%
\end{align*}
una vez m\'{a}s hemos utilizado la definici\'{o}n $\Gamma\left(  z\right)
\equiv\int_{0}^{\infty}e^{-t}t^{z-1}\mathrm{d}t\equiv\left(  z-1\right)  !$ .
Por lo tanto la serie en cuesti\'{o}n queda como
\[
f(x)=x^{\nu}=\Gamma\left(  \nu+\alpha+1\right)  \Gamma\left(  \nu+1\right)
\sum_{n=0}^{\infty}(-1)^{n}\frac{L_{n}^{\alpha}(x)}{\Gamma\left(
n+\alpha+1\right)  \Gamma\left(  \nu-n+1\right)  }%
\]
en particular si $\nu=p,$ un entero positivo, la serie se termina en un
n\'{u}mero finito de t\'{e}rminos
\[
f(x)=x^{p}=\Gamma\left(  p+\alpha+1\right)  p!\sum_{n=0}^{p}(-1)^{n}%
\frac{L_{n}^{\alpha}(x)}{\Gamma\left(  n+\alpha+1\right)  (p-n)!}%
\]

\begin{center}
\textbf{Funciones de Bessel}
\end{center}

La Ecuaci\'{o}n de Bessel es
\[
x^{2}y^{\prime\prime}+xy^{\prime}+\left(  x^{2}-k^{2}\right)  y=0;\qquad
k\in\Re
\]
obviamente $x=0$ es una singularidad regular, por lo tanto el m\'{e}todo de
Frobenius nos permite afirmar que si $x=x_{0}$ corresponde a un polo regular
de la ecuaci\'{o}n
\[
x^{2}y^{\prime\prime}+x\tilde{P}\left(  x\right)  y^{\prime}+\tilde{Q}\left(
x\right)  y=0;
\]
la soluci\'{o}n vendr\'{a} expresada de la forma
\[
y(x)=\left(  x-x_{0}\right)  ^{r}\sum_{n=0}^{\infty}a_{n}\left(
x-x_{0}\right)  ^{n}
\]
con $r$ real y determinado a trav\'{e}s de las ra\'{\i}ces de la ecuaci\'{o}n
indicadora
\[
r^{2}+\left(  \tilde{P}(x_{0})-1\right)  r+\tilde{Q}(x_{0})=0
\]
y donde $\tilde{P}\left(  x\right)  $ y $\tilde{Q}\left(  x\right)  $ son
funciones anal\'{\i}ticas en el entorno de $x=x_{0}$ y por lo tanto
\[
\tilde{P}(x_{0})=\sum_{n=0}^{\infty}b_{n}\left(  x-x_{0}\right)  ^{n}%
\quad\wedge\quad\tilde{Q}(x_{0})=\sum_{n=0}^{\infty}c_{n}\left(
x-x_{0}\right)  ^{n}
\]
Para la Ecuaci\'{o}n de Bessel
\[
\tilde{P}\left(  x\right)  =1\Rightarrow b_{0}=1\quad\wedge\quad\tilde
{Q}(x)=\left(  x^{2}-k^{2}\right)  \Rightarrow c_{0}=-k^{2};\quad c_{2}=1
\]
los dem\'{a}s coeficientes $b$'s y $c$'s se anulan. La ecuaci\'{o}n indicadora
y sus ra\'{\i}ces quedan como
\[
m\left(  m-1\right)  +m-k^{2}=0\quad\Rightarrow\quad m^{2}=k^{2}%
\quad\Rightarrow\quad r_{1,2}=\pm k
\]
\textit{\ }Donde, para $r=k$ proponemos
\[
y_{1}(x)=x^{k}\sum_{n=0}^{\infty}a_{n}x^{n}
\]
Al hacer las cuentas
\begin{align*}
\left(  x^{2}-k^{2}\right)  y_{1}(x)  & =x^{k}\sum_{n=2}^{\infty}a_{n-2}%
x^{n}-x^{k}\sum_{n=0}^{\infty}k^{2}\ a_{n}x^{n}\\
xy_{1}^{\prime}(x)  & =x^{k}\sum_{n=0}^{\infty}\left(  k+p\right)  a_{n}%
x^{n}\\
x^{2}y_{1}^{\prime\prime}(x)  & =x^{k}\sum_{n=0}^{\infty}\left(  k+p\right)
\left(  k+p-1\right)  a_{n}x^{n}%
\end{align*}
la ecuaci\'{o}n de Bessel queda como
\begin{align*}
\sum_{n=0}^{\infty}\left[  \left(  k+n\right)  \left(  k+n-1\right)  +\left(
k+n\right)  -k^{2}\right]  a_{n}x^{n}+\sum_{n=2}^{\infty}a_{n-2}x^{n}  & =0\\
\left(  2n+1\right)  a_{1}x+\sum_{n=2}^{\infty}\left[  k\left(  2n+k\right)
a_{k}+a_{n-2}\right]  x^{n}  & =0
\end{align*}
y por consiguiente obtenemos la relaci\'{o}n de recurrencia
\[
a_{n}=-\frac{a_{n-2}}{n\left(  2k+n\right)  }
\]
donde es claro que $a_{1}=0.$ Adicionalmente, si suponemos
\[
a_{0}=\frac1{2^{k}\ \Gamma\left(  k+1\right)  }
\]
tendremos
\begin{align*}
a_{1}  & =a_{3}=a_{5}=\cdots=0\\
a_{2}  & =-\frac{a_{0}}{2\left(  2k+2\right)  }\\
a_{4}  & =\frac{a_{0}}{2\cdot4\left(  2k+2\right)  \left(  2k+4\right)  }\\
& \vdots\\
a_{2n}  & =\left(  -1\right)  ^{n}\frac{a_{0}}{2^{2n}\ n!\left(  k+1\right)
\left(  k+2\right)  \cdots\left(  k+n\right)  }%
\end{align*}
Por lo tanto, la primera de las soluciones ser\'{a}%

\[
J_{k}(x)=\sum_{n=0}^{\infty}\frac{\left(  -1\right)  ^{n}}{\Gamma\left(
n+1\right)  \Gamma\left(  n+k+1\right)  }\left(  \frac{x}{2}\right)  ^{2n+k}%
\]
la \textit{Funci\'{o}n de Bessel, de orden k de primera especie.}%

%TCIMACRO{\FRAME{ftbpF}{4.5238in}{3.3987in}{0pt}{}{}{bessel1.jpg}%
%{\special{ language "Scientific Word";  type "GRAPHIC";
%maintain-aspect-ratio TRUE;  display "USEDEF";  valid_file "F";
%width 4.5238in;  height 3.3987in;  depth 0pt;  original-width 9.9895in;
%original-height 7.4893in;  cropleft "0";  croptop "1";  cropright "1";
%cropbottom "0";  filename 'bessel1.JPG';file-properties "XNPEU";}}}%
%BeginExpansion
\begin{figure}
[ptb]
\begin{center}
\includegraphics[
natheight=7.489300in,
natwidth=9.989500in,
height=3.3987in,
width=4.5238in
]%
{bessel1.jpg}%
\end{center}
\end{figure}
%EndExpansion

Si $k=0$ entonces
\[
J_{0}(x)=\sum_{n=0}^{\infty}\frac{\left(  -1\right)  ^{n}}{\left(  n!\right)
^{2}}\left(  \frac x2\right)  ^{2n}
\]
Para el caso particular de $k=m$ entero positivo la funci\'{o}n de Bessel de
primera especie toma la forma de
\[
J_{m}(x)=\sum_{n=0}^{\infty}\frac{\left(  -1\right)  ^{n}}{n!\ \left(
n+m\right)  !}\left(  \frac x2\right)  ^{2n+m}
\]

Para encontrar la segunda soluci\'{o}n linealmente independiente de la
ecuaci\'{o}n de Bessel el m\'{e}todo de Frobenius propone tres casos
dependiendo el valor de $k$%
\[
\left\{
\begin{array}
[c]{l}%
r_{1}-r_{2}\neq entero\Rightarrow\ k\neq entero\\
r_{1}=r_{2}=r\Rightarrow\ k=0\\
r_{1}-r_{2}=entero\Rightarrow\ k=entero
\end{array}
\right.
\]
\textbf{Caso 1:} $r_{1}-r_{2}\neq entero\Rightarrow\ k\neq entero.$\newline La
soluci\'{o}n general ser\'{a} de la forma
\[
y(x)=C_{1}J_{k}(x)+C_{2}J_{-k}(x)
\]
donde
\[
J_{-k}(x)=\sum_{n=0}^{\infty}\frac{\left(  -1\right)  ^{n}}{\Gamma\left(
n+1\right)  \Gamma\left(  n-k+1\right)  }\left(  \frac x2\right)
^{2n-k}\qquad x>0
\]
Para $x<0$ se debe reemplazar $x^{-k}$ por $\left\|  x\right\|  ^{-k}$ .
N\'{o}tese que esta \'{u}ltima expresi\'{o}n tambi\'{e}n es v\'{a}lida para
$k$ semientero, i.e. $k=n+\frac12$ .\newline \textbf{Caso 2}: $r_{1}%
=r_{2}=r\Rightarrow\ k=0.$\newline La soluci\'{o}n general ser\'{a} de la
forma
\[
K_{0}(x)=\sum_{n=0}^{\infty}\tilde{a}_{n}x^{n}+J_{0}(x)\ \ln x
\]
y los coeficientes $\tilde{a}_{n}$ se encuentran mediante el tradicional
m\'{e}todo de sustituirlos en la ecuaci\'{o}n de Bessel para $k=0$%
\[
xy^{\prime\prime}+y^{\prime}+xy=0;
\]
De donde se obtiene
\begin{align*}
xK_{0}(x)  & =\sum_{n=0}^{\infty}\tilde{a}_{n}x^{n+1}+xJ_{0}(x)\ \ln
x=\sum_{n=3}^{\infty}\tilde{a}_{n-2}x^{n-1}+xJ_{0}(x)\ \ln x\\
K_{0}^{\prime}(x)  & =\sum_{n=0}^{\infty}n\tilde{a}_{n}x^{n-1}+\left(
J_{0}(x)\ \ln x\right)  ^{\prime}=\sum_{n=1}^{\infty}n\tilde{a}_{n}%
x^{n-1}+J_{0}^{\prime}(x)\ \ln x+\frac{J_{0}(x)}x\\
xK_{0}^{\prime\prime}(x)  & =\sum_{n=2}^{\infty}n\left(  n-1\right)  \tilde
{a}_{n}x^{n-1}+xJ_{0}^{\prime\prime}(x)\ \ln x+2J_{0}^{\prime}(x)-\frac
{J_{0}(x)}x
\end{align*}
y por lo tanto
\[
\tilde{a}_{1}+4\tilde{a}_{2}x+\sum_{n=3}^{\infty}\left[  n^{2}\tilde{a}%
_{n}+\tilde{a}_{n-2}\right]  x^{n-1}+\left[  \underbrace{xJ_{0}^{\prime\prime
}+J_{0}^{\prime}+xJ_{0}}_{=\ 0}\right]  \ln x+2J_{0}^{\prime}(x)=0
\]
Acomodando y derivando la expresi\'{o}n para $J_{0}$ tendremos
\[
\tilde{a}_{1}+4\tilde{a}_{2}x+\sum_{n=3}^{\infty}\left[  n^{2}\tilde{a}%
_{n}+\tilde{a}_{n-2}\right]  x^{n-1}=-2J_{0}^{\prime}(x)=2\sum_{n=1}^{\infty
}\frac{2n}{2^{2n-1}}\frac{\left(  -1\right)  ^{n+1}}{\left(  n!\right)  ^{2}%
}x^{2n-1}
\]
Ahora multiplicando la expresi\'{o}n por $x$ y separando las sumatorias en sus
t\'{e}rminos pares e impares, tendremos
\begin{align*}
\tilde{a}_{1}x+\sum_{n=1}^{\infty}\left[  \left(  2n+1\right)  ^{2}\tilde
{a}_{2n+1}+\tilde{a}_{2n-1}\right]  x^{2n+1}  & =0\\
\sum_{n=2}^{\infty}\left[  \left(  2n\right)  ^{2}\tilde{a}_{2n}+\tilde
{a}_{2n-2}\right]  x^{2n}+4\tilde{a}_{2}x^{2}  & =x^{2}+\sum_{n=1}^{\infty
}\left(  -1\right)  ^{n+1}\frac{2n}{2^{2n}\left(  n!\right)  ^{2}}x^{2n}%
\end{align*}
Por lo cual $\tilde{a}_{1}=\tilde{a}_{3}=\tilde{a}_{5}=\cdots=0$ mientras que
\[
4\tilde{a}_{2}=1;\qquad\left(  2n\right)  ^{2}\tilde{a}_{2n}+\tilde{a}%
_{2n-2}=\left(  -1\right)  ^{n+1}\frac{2n}{2^{2n}\left(  n!\right)  ^{2}%
}\qquad n>1
\]
De esta forma los coeficientes quedan como:
\begin{align*}
\tilde{a}_{2}  & =\frac1{2^{2}}\\
\tilde{a}_{4}  & =-\frac1{2^{2}\cdot4^{2}}\left(  1+\frac12\right)
=-\frac1{2^{4}\cdot\left(  2!\right)  ^{2}}\left(  1+\frac12\right) \\
& \vdots\\
\tilde{a}_{2n}  & =\frac{\left(  -1\right)  ^{n+1}}{2^{2n}\ \left(  n!\right)
^{2}}\left\{  1+\frac12+\frac13+\frac14+\cdots+\frac1k\right\}
\end{align*}
La expresi\'{o}n para la soluci\'{o}n general de la ecuaci\'{o}n de Bessel
para $k=0$ ser\'{a}
\[
K_{0}(x)=\sum_{n=0}^{\infty}\frac{\left(  -1\right)  ^{n+1}}{\left(
n!\right)  ^{2}}\left\{  1+\frac12+\frac13+\frac14+\cdots+\frac1k\right\}
\left(  \frac x2\right)  ^{2n}+J_{0}(x)\ \ln x
\]
En F\'{\i}sica, es costumbre expresar esta soluci\'{o}n de una forma
equivalente pero ligeramente diferente:
\[
Y_{0}(x)=-\frac2\pi\sum_{n=0}^{\infty}\frac{\left(  -1\right)  ^{n}}{\left(
n!\right)  ^{2}}\left\{  1+\frac12+\frac13+\cdots+\frac1k\right\}  \left(
\frac x2\right)  ^{2n}+\frac2\pi J_{0}(x)\ \left[  \ln\frac x2+\gamma\right]
\]
donde, una vez m\'{a}s, $\gamma=0.577215664901\cdots$ es la constante de
Euler-Mascheroni. \newline \textbf{Caso 3}: $r_{1}-r_{2}=entero\Rightarrow
\ k=entero.$\newline La soluci\'{o}n general ser\'{a} de la forma
\[
K_{k}(x)=\sum_{n=0}^{\infty}\tilde{a}_{n}x^{k+n}+CJ_{n}(x)\ \ln x
\]
Procediendo de forma equivalente a la situaci\'{o}n anterior tenemos que la
soluci\'{o}n general podr\'{a} expresarse (luego de una laboriosa faena) como
\begin{align*}
K_{k}(x)  & =-\frac12\sum_{n=0}^{k-1}\frac{\left(  k-n-1\right)  !}{n!}\left(
\frac x2\right)  ^{2n-k}-\frac{H_{k}}{2k!}\left(  \frac x2\right)  ^{k}-\\
& -\frac12\sum_{n=1}^{\infty}\frac{\left(  -1\right)  ^{n}\left[
H_{n}+H_{n+k}\right]  }{n!\left(  k+n\right)  !}\left(  \frac x2\right)
^{2n+k}+J_{k}(x)\ \ln x
\end{align*}
Y finalmente la \textit{Funci\'{o}n de Bessel de orden }$k$ \textit{de segunda
especie} o \textit{Funci\'{o}n de Neumann}
\begin{align*}
Y_{k}(x)  & =-\frac1\pi\sum_{n=0}^{k-1}\frac{\left(  k-n-1\right)  !}{\left(
n!\right)  ^{2}}\left(  \frac x2\right)  ^{2n-k}-\frac{H_{k}}{\pi k!}\left(
\frac x2\right)  ^{k}-\\
& -\frac1\pi\sum_{n=1}^{\infty}\frac{\left(  -1\right)  ^{n}\left[
H_{n}+H_{n+k}\right]  }{n!\left(  k+n\right)  !}\left(  \frac x2\right)
^{2n+k}+\frac2\pi J_{k}(x)\ \left[  \ln\frac x2+\gamma\right]
\end{align*}
En ambos casos
\[
H_{n}=1+\frac12+\frac13+\cdots+\frac1n
\]
M\'{a}s a\'{u}n
\begin{align*}
Y_{k}(x)  & =\frac2\pi J_{k}(x)\ \ln\frac x2-\frac1\pi\sum_{n=0}^{k-1}%
\frac{\left(  k-n-1\right)  !}{\left(  n!\right)  ^{2}}\left(  \frac
x2\right)  ^{2n-k}\\
& -\frac1\pi\sum_{n=1}^{\infty}\frac{\left(  -1\right)  ^{n}}{n!\left(
k+n\right)  !}\left(  \frac x2\right)  ^{2n+k}\left[  \psi(n+1)+\psi
(n+k+1)\right]
\end{align*}
donde $\psi(n)=\dfrac{\Gamma^{\prime}(n)}{\Gamma(n)}$ es la funci\'{o}n
Digamma con
\begin{align*}
\psi(n+1)  & =-\gamma+1+\frac12+\frac13+\cdots+\frac1n\\
\psi(1)  & =-\gamma
\end{align*}
Tambi\'{e}n es costumbre definir la funci\'{o}n de Bessel de segunda especie
en terminos de las de primera especie
\[
N_{k}(x)=Y_{k}(x)=\frac{J_{k}(x)\cos k\pi-J_{-k}(x)}{\operatorname*{sen}k\pi}
\]
N\'{o}tese que para $k=m$ entero, aparentemente no esta definida. Pero,
aplicando la regla de L'Hospital
\begin{align*}
N_{m}(x)  & =\left.  \frac{\dfrac{\mathrm{d}}{\mathrm{d}k}\left[  J_{k}(x)\cos
k\pi-J_{-k}(x)\right]  }{\dfrac{\mathrm{d}}{\mathrm{d}k}\left[
\operatorname*{sen}k\pi\right]  }\right|  _{k=m}\\
& =\left.  \frac{-\pi J_{n}(x)\operatorname*{sen}n\pi+\left\{  \cos n\pi
\dfrac{\mathrm{d}}{\mathrm{d}k}J_{k}(x)-\dfrac{\mathrm{d}}{\mathrm{d}k}%
J_{-k}(x)\right\}  }{\pi\cos n\pi}\right|  _{k=m}\\
& =\frac1\pi\left\{  \dfrac{\mathrm{d}}{\mathrm{d}k}J_{k}(x)-\left(
-1\right)  ^{n}\dfrac{\mathrm{d}}{\mathrm{d}k}J_{-k}(x)\right\}  _{k=m}%
\end{align*}

De este modo, la soluci\'{o}nes generales para la ecuaci\'{o}n de Bessel, se
expresan seg\'{u}n el caso en
\begin{align*}
Z_{k}(x)  & =C_{1}J_{k}(x)+C_{2}J_{-k}(x);\qquad k\neq entero\\
\tilde{Z}_{k}(x)  & =C_{1}J_{k}(x)+C_{2}Y_{k}(x);\qquad k=0\quad\vee\quad
entero
\end{align*}
La funciones $Z_{k}(x)$ y $\tilde{Z}_{k}(x)$ se denominan \textit{Funciones
Cil\'{i}ndricas de orden }$k$%

%TCIMACRO{\FRAME{ftbpF}{4.5238in}{3.3987in}{0pt}{}{}{bessel2.jpg}%
%{\special{ language "Scientific Word";  type "GRAPHIC";
%maintain-aspect-ratio TRUE;  display "USEDEF";  valid_file "F";
%width 4.5238in;  height 3.3987in;  depth 0pt;  original-width 9.9895in;
%original-height 7.4893in;  cropleft "0";  croptop "1";  cropright "1";
%cropbottom "0";  filename 'bessel2.JPG';file-properties "XNPEU";}}}%
%BeginExpansion
\begin{figure}
[ptb]
\begin{center}
\includegraphics[
natheight=7.489300in,
natwidth=9.989500in,
height=3.3987in,
width=4.5238in
]%
{bessel2.jpg}%
\end{center}
\end{figure}
%EndExpansion

\begin{center}
\textbf{Propiedades de las Funciones de Bessel}
\end{center}

\textbf{Otras Formas de la Ecuaci\'{o}n de Bessel}\newline Haciendo los
cambios de variables correspondientes llegamos a
\[
u^{\prime\prime}(x)+\frac{1-2\alpha}xu^{\prime}(x)+\left[  \left(  \beta
\nu\ x^{\nu-1}\right)  ^{2}+\frac{\alpha^{2}-k^{2}\nu^{2}}{x^{2}}\right]
u(x)=0
\]
donde
\[
u(x)=x^{\alpha}Z_{k}(\beta x^{\nu})
\]
o tambi\'{e}n
\[
u^{\prime\prime}(x)+\alpha x^{\nu}\ u(x)=0
\]
con
\[
u(x)=\sqrt{x}Z_{\frac1{\nu+2}}\left(  \frac{2\sqrt{\alpha}}{\nu+2}%
x^{1+\frac\nu2}\right)
\]

\begin{center}
\textbf{Relaciones de Recurrencia:}
\end{center}

Las funciones de Bessel tienen las siguientes relaciones de recurrencia
\begin{align*}
xJ_{k+1}(x)-2k\ J_{k}(x)+xJ_{k-1}(x)  & =0\\
J_{k+1}(x)+2J_{k}^{\prime}(x)-J_{k-1}(x)  & =0
\end{align*}
Para demostrar estas relaciones partimos por demostrar la siguiente identidad
\begin{align*}
\left[  x^{k}J_{k}(x)\right]  ^{\prime}  & =x^{k}J_{k-1}(x)\\
\left[  x^{-k}J_{k}(x)\right]  ^{\prime}  & =-x^{-k}J_{k+1}(x)
\end{align*}
De la expresi\'{o}n para $J_{k}(x)$ se obtiene
\begin{align*}
\left[  \sum_{n=0}^{\infty}\frac{\left(  -1\right)  ^{n}}{\Gamma\left(
n+1\right)  \Gamma\left(  n+k+1\right)  }\left(  \frac x2\right)
^{2n+2k}\right]  ^{\prime}  & =\sum_{n=0}^{\infty}\frac{\left(  -1\right)
^{n}2\left(  n+k\right)  x^{2n+2k-1}}{2^{2n+k}\Gamma\left(  n+1\right)
\Gamma\left(  n+k+1\right)  }\\
& =x^{k}\sum_{n=0}^{\infty}\frac{\left(  -1\right)  ^{n}x^{2n+\left(
k-1\right)  }}{2^{2n+\left(  k-1\right)  }\Gamma\left(  n+1\right)
\Gamma\left(  n+k\right)  }\\
& =x^{k}J_{k-1}(x)
\end{align*}
Unos cambios apropiados nos llevan a demostrar las segunda de las relaciones y
al desarrollar las derivadas
\begin{align*}
\left[  x^{k}J_{k}(x)\right]  ^{\prime}  & =kx^{k-1}J_{k}(x)+x^{k}%
J_{k}^{\prime}(x)=x^{k}J_{k-1}(x)\\
\left[  x^{-k}J_{k}(x)\right]  ^{\prime}  & =-kx^{-k-1}J_{k}(x)+x^{-k}%
J_{k}^{\prime}(x)=-x^{-k}J_{k+1}(x)
\end{align*}
Por lo cual
\begin{align*}
kJ_{k}(x)+xJ_{k}^{\prime}(x)  & =xJ_{k-1}(x)\\
-kJ_{k}(x)+xJ_{k}^{\prime}(x)  & =-xJ_{k+1}(x)
\end{align*}
Al sumar y restar miembro a miembro obtenemos las relaciones de recurrencia.
Es obvia la importancia que adquieren $J_{1}(x)$ y $J_{0}(x)$ para generar el
resto de las funciones de Bessel.

\begin{center}
\textbf{Funciones de Bessel y Funciones Elementales}
\end{center}

Las funciones de Bessel de \'{o}rden semientero, $k=\frac12$ se expresa como
\[
J_{1/2}(x)=\sqrt{\frac x2}\sum_{n=0}^{\infty}\frac{\left(  -1\right)  ^{n}%
}{\Gamma\left(  n+1\right)  \Gamma\left(  n+\frac32\right)  }\left(  \frac
x2\right)  ^{2n}
\]
pero como
\[
\Gamma\left(  n+\frac32\right)  =\left\{  \frac32\cdot\frac52\cdot\cdots
\frac{2n+1}2\right\}  =\Gamma\left(  \frac32\right)  \frac{1\cdot3\cdot
5\cdots\left(  2n+1\right)  }{2^{n}}
\]
se encuentra que
\begin{align*}
J_{1/2}(x)  & =\sqrt{\frac x2}\sum_{n=0}^{\infty}\frac{\left(  -1\right)
^{n}}{2^{n}\ n!\Gamma\left(  \frac32\right)  1\cdot3\cdot5\cdots\left(
2n+1\right)  }x^{2n}\\
& =\frac x{\sqrt{2x}\Gamma\left(  \frac32\right)  }\left\{  1-\frac{x^{2}%
}{2\cdot3}+\frac{x^{4}}{2\cdot4\cdot3\cdot5}-\frac{x^{6}}{2\cdot4\cdot
6\cdot3\cdot5\cdot7}+\cdots\right\} \\
& =\frac1{\sqrt{2x}\Gamma\left(  \frac32\right)  }\left\{  1-\frac{x^{3}}%
{3!}+\frac{x^{5}}{5!}-\frac{x^{7}}{7!}+\cdots\right\}  =\frac1{\sqrt{2x}%
\Gamma\left(  \frac32\right)  }\operatorname*{sen}x
\end{align*}
Finalmente, y otra vez invocando a las propiedades de la funci\'{o}n Gamma:
$\Gamma\left(  \frac32\right)  =\frac{\sqrt{\pi}}2$%
\[
J_{1/2}(x)=\sqrt{\frac2{\pi x}}\operatorname*{sen}x
\]
Equivalentemente se puede demostrar que
\[
J_{-1/2}(x)=\sqrt{\frac2{\pi x}}\cos x
\]
y ahora utilizando las relaciones de recurrencia tendremos que
\begin{align*}
J_{3/2}(x)  & =-J_{-1/2}(x)+\frac1xJ_{1/2}(x)\\
& =\sqrt{\frac2{\pi x}}\left[  \frac{\operatorname*{sen}x}x-\cos x\right]
\end{align*}
As\'{\i} mismo
\begin{align*}
J_{5/2}(x)  & =-J_{1/2}(x)+\frac3xJ_{3/2}(x)\\
& =\sqrt{\frac2{\pi x}}\left[  \frac{3\operatorname*{sen}x}{x^{2}}-\frac{3\cos
x}x-\operatorname*{sen}x\right]
\end{align*}
En general
\begin{align*}
J_{n+\frac12}(x)  & =\left(  -1\right)  ^{n}\sqrt{\frac2\pi}x^{n+\frac12}%
\frac{\mathrm{d}^{n}}{\left(  x\mathrm{d}x\right)  ^{n}}\left(  \frac
{\operatorname*{sen}x}x\right)  \qquad n=1,2,3,\cdots\\
J_{n+\frac12}(x)  & =\sqrt{\frac2\pi}x^{n+\frac12}\frac{\mathrm{d}^{n}%
}{\left(  x\mathrm{d}x\right)  ^{n}}\left(  \frac{\cos x}x\right)  \qquad
n=-1,-2,-3,\cdots
\end{align*}
Las funciones de Bessel de \'{o}rden semientero son las \'{u}nicas funciones
de Bessel que pueden ser expresadas en t\'{e}rminos de funciones elementales.

\begin{center}
\textbf{Reflexi\'{o}n:}
\end{center}

Las funciones de Bessel cumplen con
\[
J_{-m}(x)=\left(  -1\right)  ^{m}J_{m}(x)
\]
Para el caso $k=m$ entero positivo la Funci\'{o}n de Bessel de primera especie
toma la forma de
\[
J_{m}(x)=\sum_{n=0}^{\infty}\frac{\left(  -1\right)  ^{n}}{n!\ \left(
n+m\right)  !}\left(  \frac x2\right)  ^{2n+m}
\]
Si $k=-m$ es un entero negativo los primeros $m$ t\'{e}rminos de la serie
anterior se anulan ya que $\Gamma(n)\rightarrow\infty$ para $n=-1,-2,-3,\cdots
$ y la serie se arma como
\begin{align*}
J_{-m}(x)  & =\sum_{n=m}^{\infty}\frac{\left(  -1\right)  ^{n}}{n!\ \left(
n-m\right)  !}\left(  \frac x2\right)  ^{2n+m}=\sum_{l=0}^{\infty}%
\frac{\left(  -1\right)  ^{l+m}}{\left(  l+m\right)  !\ l!}\left(  \frac
x2\right)  ^{2l+m}\\
J_{-m}(x)  & =\left(  -1\right)  ^{m}J_{m}(x)
\end{align*}

\textbf{Funci\'{o}n Generatriz}\newline La funci\'{o}n generatriz para las
Funciones de Bessel es
\[
\mathcal{B}(x,t)=\mathrm{e}^{\frac x2\left(  t-\frac1t\right)  }
\]
desarrollando las dos series para las exponenciales
\begin{align*}
\mathrm{e}^{\dfrac{xt}2}  & =1+\frac x2t+\frac x{2^{2}2!}t^{2}+\cdots
+\frac{x^{n}}{2^{n}n!}t^{n}+\cdots\\
\mathrm{e}^{\dfrac x{2t}}  & =1-\frac x2t^{-1}+\frac x{2^{2}2!}t^{-2}%
+\cdots+\frac{\left(  -1\right)  ^{n}x^{n}}{2^{n}n!}t^{-n}+\cdots
\end{align*}
Por lo tanto multiplicando ambas series
\[
\mathcal{B}(x,t)=\mathrm{e}^{\frac x2\left(  t-\frac1t\right)  }=\left\{
\sum_{n=0}^{\infty}\frac{x^{n}}{2^{n}n!}t^{n}\right\}  \left\{  \sum
_{n=0}^{\infty}\frac{\left(  -1\right)  ^{n}x^{n}}{2^{n}n!}t^{-n}\right\}
=\sum_{n=-\infty}^{\infty}J_{n}(x)\ t^{n}
\]

\begin{center}
\textbf{Representaci\'{o}n Integral para las Funciones de Bessel}
\end{center}

En la expresi\'{o}n anterior para la funci\'{o}n generatriz se realiza el
siguiente cambio de varible $t=\mathrm{e}^{i\theta}$ de este modo
\[
\mathrm{e}^{\frac x2\left(  t-\frac1t\right)  }=\mathrm{e}%
^{ix\operatorname*{sen}\theta}=\cos\left(  x\operatorname*{sen}\theta\right)
+i\operatorname*{sen}\left(  x\operatorname*{sen}\theta\right)
\]
y por lo tanto
\[
\cos\left(  x\operatorname*{sen}\theta\right)  +i\operatorname*{sen}\left(
x\operatorname*{sen}\theta\right)  =\sum_{n=-\infty}^{\infty}J_{n}(x)\ \left[
\cos\left(  n\theta\right)  +i\operatorname*{sen}\left(  n\theta\right)
\right]
\]
igualando partes reales e imaginarias y recordando que $J_{-m}(x)=\left(
-1\right)  ^{m}J_{m}(x)$, para anular los t\'{e}rminos impares en la serie de
la parte real y los pares en la de la parte imaginaria, podemos escribir
\begin{align*}
\cos\left(  x\operatorname*{sen}\theta\right)   & =J_{0}(x)+2\sum
_{n=1}^{\infty}J_{2n}(x)\cos\left(  2n\theta\right) \\
\operatorname*{sen}\left(  x\operatorname*{sen}\theta\right)   & =2\sum
_{n=0}^{\infty}J_{2n+1}(x)\operatorname*{sen}\left(  \left[  2n+1\right]
\theta\right)
\end{align*}
Multiplicando miembro a miembro en la primera de ellas por $\cos\left(
2k\theta\right)  $ (y por $\cos\left(  \left[  2k+1\right]  \theta\right)  $ )
y la segunda por $\operatorname*{sen}\left(  \left[  2k+1\right]
\theta\right)  $ (y por $\operatorname*{sen}\left(  2k\theta\right)  $).
Integrando (en $0\leq\theta\leq\pi$), tambi\'{e}n miembro a miembro y
t\'{e}rmino por t\'{e}rmino en las series, se obtienen
\begin{align*}
J_{2n}(x)  & =\frac1\pi\int_{0}^{\pi}\cos\left(  x\operatorname*{sen}%
\theta\right)  \cos\left(  2n\theta\right)  \ \mathrm{d}\theta\\
0  & =\frac1\pi\int_{0}^{\pi}\cos\left(  x\operatorname*{sen}\theta\right)
\cos\left(  \left[  2n+1\right]  \theta\right)  \ \mathrm{d}\theta\\
J_{2n+1}(x)  & =\frac1\pi\int_{0}^{\pi}\operatorname*{sen}\left(
x\operatorname*{sen}\theta\right)  \operatorname*{sen}\left(  \left[
2n+1\right]  \theta\right)  \ \mathrm{d}\theta\\
0  & =\frac1\pi\int_{0}^{\pi}\operatorname*{sen}\left(  x\operatorname*{sen}%
\theta\right)  \operatorname*{sen}\left(  2n\theta\right)  \ \mathrm{d}\theta
\end{align*}
Sumando miembro a miembro primera con cuarta y segunda con tercera tendremos
la expresi\'{o}n integral para las funciones de Bessel
\[
J_{n}(x)=\frac1\pi\int_{0}^{\pi}\cos\left(  \cos\left(  n\theta\right)
-x\operatorname*{sen}\theta\right)  \ \mathrm{d}\theta
\]
ya que todos sabemos que
\[
\cos\left(  n\theta-x\operatorname*{sen}\theta\right)  =\cos\left(
2n\theta\right)  \cos\left(  x\operatorname*{sen}\theta\right)
+\operatorname*{sen}\left(  2n\theta\right)  \operatorname*{sen}\left(
x\operatorname*{sen}\theta\right)
\]

\begin{center}
\textbf{Ortogonalidad de las Funciones de Bessel}

\textbf{y Series de Bessel-Fourier}
\end{center}

\textbf{Ortogonalidad:}\newline Haciendo el caso particular de $\alpha=0$ y
$\nu=1$ en la primera de las expresiones equivalentes para la ecuaci\'{o}n de
Bessel, tendremos
\[
u^{\prime\prime}(x)+\frac{1}{x}u^{\prime}(x)+\left[  \beta^{2}-\frac{k^{2}%
}{x^{2}}\right]  u(x)=0
\]
donde
\[
u(x)=J_{k}(\beta x)
\]
multiplicando por $x$ la ecuacion diferencial puede ser reescrita como
\[
\left[  xJ_{k}^{\prime}(\beta x)\right]  ^{\prime}+\left[  \beta^{2}%
x-\frac{k^{2}}{x}\right]  J_{k}(\beta x)=0
\]
suponiendo $k$ real y positivo, planteamos la ecuaci\'{o}n para dos
\'{i}ndices diferentes $\beta_{1}$ y $\beta_{2}$ por lo tanto quedan como
\begin{align*}
\left[  xJ_{k}^{\prime}(\beta_{1}x)\right]  ^{\prime}+\left[  \beta_{1}%
^{2}x-\frac{k^{2}}{x}\right]  J_{k}(\beta_{1}x)  & =0\\
\left[  xJ_{k}^{\prime}(\beta_{2}x)\right]  ^{\prime}+\left[  \beta_{2}%
^{2}x-\frac{k^{2}}{x}\right]  J_{k}(\beta_{2}x)  & =0
\end{align*}
Multiplicando apropiadamente por $J_{k}(\beta_{1}x)$ y $J_{k}(\beta_{2}x),$
Integrando y restando miembro a miembro tendremos que
\begin{align*}
\left(  \beta_{2}^{2}-\beta_{1}^{2}\right)  \int_{0}^{1}xJ_{k}(\beta
_{1}x)J_{k}(\beta_{2}x)\mathrm{d}x  & =\int_{0}^{1}\left\{  J_{k}(\beta
_{2}x)\left[  xJ_{k}^{\prime}(\beta_{1}x)\right]  ^{\prime}-J_{k}(\beta
_{1}x)\left[  xJ_{k}^{\prime}(\beta_{2}x)\right]  ^{\prime}\right\}
\mathrm{d}x\\
& =\int_{0}^{1}\left[  J_{k}(\beta_{2}x)xJ_{k}^{\prime}(\beta_{1}%
x)-J_{k}(\beta_{1}x)xJ_{k}^{\prime}(\beta_{2}x)\right]  ^{\prime}%
\mathrm{d}x\\
& =\left.  J_{k}(\beta_{2}x)xJ_{k}^{\prime}(\beta_{1}x)-J_{k}(\beta
_{1}x)xJ_{k}^{\prime}(\beta_{2}x)\right|  _{x=0}^{x=1}%
\end{align*}
para $\beta_{i}$ las ra\'{i}ces de los polinomios de Bessel, i.e. $J_{k}%
(\beta_{i})=0$ podemos deducir que las funciones de Bessel son ortogonales
\[
\left(  \beta_{i}^{2}-\beta_{j}^{2}\right)  \int_{0}^{1}xJ_{k}(\beta
_{i}x)J_{k}(\beta_{j}x)\mathrm{d}x\propto\delta_{ij}%
\]
M\'{a}s a\'{u}n partiendo de la ecuaci\'{o}n de Bessel original se puede
llegar a
\[
\left\|  J_{k}(\beta x)\right\|  ^{2}=\frac{1}{2}\left[  J_{k}^{\prime}%
(\beta)\right]  ^{2}+\frac{\beta^{2}-k^{2}}{2\beta^{2}}\left[  J_{k}%
(\beta)\right]  ^{2}%
\]
\end{document}