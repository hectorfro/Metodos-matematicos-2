\documentclass[spanish,notitlepage,letterpaper,11pt]{article}
\usepackage{graphicx}
\usepackage{amsmath}
\usepackage[spanish]{babel}
\usepackage[ansinew]{inputenc}
\usepackage{fancyhdr}
\usepackage{fancyhdr}
\usepackage{amsfonts}
\usepackage{amssymb}

\hyphenation{La-ti-noa-me-ri-ca-na Fa-cul-tad Ren-di-mien-to}
\pagestyle{fancy}
\lhead{}
\chead{}
\cfoot{}
\rhead{\bfseries Series y dem�s hierbas}
\lhead{\bfseries Tareas de M�todos Matem�ticos 2}
\lfoot{\it Luis A. N��ez  }
\cfoot{ Universidad de Los Andes, M�rida, Venezuela}
\rfoot{\thepage}
\voffset = -0.25in
\textheight = 8.0in
\textwidth = 6.5in
\oddsidemargin = 0.in
\headheight = 20pt
\headwidth = 6.5in
\renewcommand{\headrulewidth}{0.5pt}
\renewcommand{\footrulewidth}{0,5pt}

\begin{document}

\begin{center}
\textbf{M�todos Matem�ticos 2}

\textbf{Tarea 3}

\textbf{Series, Series, Series}

\textbf{Fecha de entrega 18 abril 2006}
\end{center}

\begin{enumerate}

\item  Muestre que las siguientes series convergen al l�mite que se indican
\[
s_{1}=\sum_{k=1}^{\infty} \frac{1}{(k+1)(k+3)(k+5)}=\frac{23}{480}
\qquad \text{y} \qquad
s_{2}=\sum_{k=1}^{\infty} \frac{3k-2}{k(k+1)(k+1)}=1
\]

\item Dada la siguiente relaci�n de recurrencia $ s_{n+1}=\sqrt{2 + \sqrt{s_{n}}} $, muestre que 
\[
\lim_{n \rightarrow \infty} s_{n} = s_{0} \qquad \text{con } s_{0} \text{ ra�z de polinomio } 
s^4 -4s^2 -s +4 = 0.
\]
\item Para cu�les valores de las constantes $\alpha$ y $\beta$ converge al siguiente serie
\[
s_{3}=\sum_{k=1}^{\infty} \frac{1}{ n(\ln n)^{\alpha} (\ln(\ln n)) }
\]
\item En un alarde de ociosidad Ud. puede comprobar con una calculadora que 
\[ \sum_{k=1}^{100} n^{-3} = 1.202007 \] Ahora, en una arranque de inteligencia muestre que  
\[
1.202056 \leqslant \sum_{k=1}^{\infty}  n^{-3} \leqslant  1.202057
\]

\item Suponga que le interesa invertir un mill�n de Bol�vares en bonos. El inter�s es del 25 \% anual. Calcule cuanto tendr� en su cuenta, a los 25 a�os de haberlos invertidos, si los intereses se calculan y se abonan
\begin{itemize}
  \item sobre saldos semestrales
  \item sobre saldos mensuales
  \item sobre saldos diarios
\end{itemize}
\item Pruebe que la siguiente serie
\[
\sum_{k=2}^{\infty} \ln \left( \frac{n^r +(-1)^n}{n^r} \right) \quad \Rightarrow 
\left\{
\begin{array}{l }
\text{es absolutamente convergente para } r=2         \\
\text{es condicionalmente convergente para } r=1         
\end{array}
\right.
\]
 \end{enumerate}
\end{document}